Machine learning mathematical foundations and algorithms have been one of the most studied areas in both mathematics and 
computer science in the past few years, due to the constant improvement in computer features. Lots of fundamental questions 
arise: what does it mean to extract knowledge out of data?, when can we learn from the data?, how much data do we need?

One of the most researched topics in machine learning is classification problems: given a sequence of elements belonging to
a domain, which have been labeled, how could we learn from them in a way that if new samples arrive we would be able to 
label them according to the knowledge acquired?. We will specially focus on binary classification problems, those which 
only have two possible labels. We will also study, in a more empirical framework, the particular case in which one 
class has more examples than the other one: imbalance classification.

The aim of this work is both to introduce a mathematical theory for machine learning, which has in the PAC learning 
an excellent formalization, and to study and code some of the most recent algorithms that have been published in scientific
journals on the topic of imbalance classification.

\section*{Mathematical introduction}
We introduce concepts suchs as product ring and semiring of sets, and their relationship with $\sigma$-algebras. We give
the notion of measure, without distinction between $\sigma$-algebras and other sets, whereas in classic literature the latter
ones are called premeasures. We also introduce the outer measures, and we advance towards the construction of a $\sigma$-algebra
based on a premeasure over some space, and the extension of that outer measure to a measure. The result which will give us
such extension is Carathéodory's theorem.

This introduction's purpose is to provide us with basic tools to develop the following theory and to answer the question:
given a set of probabilistic spaces, does it exist a product $\sigma$-algebra space in which the product of the sets has
the product of the probabilities as measure?.

This very first part includes demonstrations for Markov and Hoeffding's inequalities, the former one based on the Hoeffding
lemma. Those will be key inequalities in our subsequent progress.

\section*{Machine learning introduction and PAC framework}
We will provide motivations for the machine learning theory, as well as basic definitions such as domain set, label set, 
true labelling, instance generation, training set, hypothesis, hypothesis' error or learning algorithm. The error of an
hypothesis respect to the true labelling function will be defined as the expectation that the two are different. The error over
a training set, as the mean of the number of instances in which it fails.

We will also describe a relation between hypothesis and the sets that those hypothesis classify as one of the classes. Thus,
we will be able to work both with hypothesis or sets. The essential algorithms we will be using will be empirical risk 
minimizers, that is, all those which try to minimize the error over a given training set.

All those concepts will sum up to the very first notion of PAC learning: we will say that a hypothesis collection is 
learnable if we can allways guarantee an small error from one of them with respect to the true labelling function, under
certain restriction over probabilistic confidence. In particular, we will prove that every finite collection of hypothesis
is learnable, and that there exist infinite learnable collections, such as rectangles hypothesis.

The PAC notions will be weakened as much as possible until we arrive to a more flexible definition of learning: APAC 
learning, which is rather similar to PAC learning, with the main difference that we can define our own error function,
and we allow the instances of both classes to suffer from overlaping (an instance could belong to both classes). As
a result, we will show that the concept of APAC implies the PAC concept under certain mild assumptions.


\section*{Uniform learning and further work}

We will define Glivenko-Cantelli classes (uniform classes) of hypothesis, and show that being Glivenko-Cantelli is an
stronger assumption than being APAC. As a result of the work developed under this chapter, we will prove again that finite
classes are learnable, but under a random error function and with worse asymptotic complexity.

In addition to this, we will show by means of the No Free Lunch Theorem, that if we only see half or less of the instances 
of a domain, that domain is not PAC learnable. Moreover, we will relate this theorem with a rather known fact: the 
bias-complexity tradeoff: the more complex the class of functions we are trying to learn, the lower the error and the
harder the learning process.

Vapnik Chervonenkis classes will also be introduced, along with the concept of shattering. This is a pure combinatorics
definition. The Sauer-Shelah lemma will gives us a relationship between the Vapnik Chervonenkis dimension and the number
of possible hypothesis in terms of their image of a training set of fixed size. 

At this stage, we will have all the pieces to prove the fundamental theorem of PAC learning, that connects statistical
concepts, such as Glivenko-Cantelli classes, combinatorial ones (the Vapnik Chervonenkis dimension) and PAC and APAC
concepts.

\section*{Introduction to the imbalance problem}

In order to study some machine learning algorithms, we will weaken our concept of learning until we reach a framework in which
we can easily explain one of data science most studied problems: imbalanced classification. We will describe the suitable
approaches to solve that problem: undersampling and oversampling with and without filtering of instances and cost-sensitive 
framework. Since most of nowadays algorithms tend to understimate the importance of the minority examples, 
we should provide a set of measurements of the quality of an algorithm based not only on the total accuracy over the training
set, but in the level of success over the minority class.

Our main focus will be on oversampling, the creation of synthetic samples belonging to the minority class. We will provide
a description of a classic method: SMOTE. Although it was first described 15 years ago, it is still a reference in the
treatment of imbalanced datasets, and a lot of other algorithms have arised by means of modification of the former one. It will be the case
of our MWMOTE algorithm.

\section*{Description of the coded algorithms}

We will analyze a series of algorithms that are not currently present in any package of \R programming language. All of
them provide an oversampling scheme, except for NEATER. A brief background theory for those algorithms will be provided.
We will start with MWMOTE, which tries to fix some of the issues of SMOTE algorithm, such as creation of new instances 
using noisy ones, or producing instances between two different minoriry class clusters.

RACOG will also be addresed, as an algorithm that approximates discrete distributions using marginal pairwise ones and 
information theory. Once the distribution has been approximated, it will extract samples following a Monte Carlo scheme
called Gibbs Sampling. wRACOG will be a modification of RACOG that aims to improve classification under a certain wrapper
method provided as parameter.

The next algorithm we will code is RWO. RWO finds its foundation on a statistics theorem that we will prove, and which guarantees
us that under asymptotic and generation scheme assumptions, our newly generated instances will preserve the original mean
and variance of the training data set.

PDFOS will be another of the algorithms selected. It is based on multivariate Gaussian kernel density estimations. We will
provide an introduction to the kernel density estimators pointing out their relationship with histograms and we will give
a measure of the error of the estimation, the Mean Integrated Squared Error. PDFOS will try to adapt the bandwith parameter
for a sum of multivariate Gaussian functions, that is, a multiplicative constant for the unbiased covariance of the minority
training samples. A gradient descendent method will be used to optimize such parameter.

The last algorithm we are going to study is NEATER, a filtering algorithm to clean up noisy instances created as a result of
oversampling. NEATER is based on game theory and Nash profile equilibriums, that guarantee that if an instance could play
two possible roles, it will tend to stabilize as one of them, given a series of payoffs based on their neighbours role.
instances.

\section*{Developed software}
An implementation for those algorithms will be made using the \R programming language combined with \texttt{C++} chunks. 
\R language provides a tradeoff between ease of use for the users and speed of the code, the second one will be used to 
speed up our algorithms when possible. Thus, we will develop a package, called \texttt{imbalance}, to fill the lack of implementation of the 
described algorithms in \R.

It is important to note not only the code, but the methodology used to develop the code: under a GPLv2 or later license,
allocated in a public Github repo and with continuous integration to provide unit testing and periodic updates of
its online web documentation page.

\section*{Experiments on small disjuncts}
Finally, we will conduct a simple experiment on small disjuncts problem, which originate as a direct result of class imbalance
problem. The experiment will somehow show that when we overcome the imbalance of the datasets, small disjuncts tend to
vanish.

We propose a simple measure of the small disjuncts which a dataset has: given a tree classifier, we will consider a small
disjunct all those leaves which label less than a given threshold examples. We will also measure the mean size of the 
leaves in terms of examples classified, sometimes called coverage.

\paragraph{Keywords}
Machine learning, PAC learning, imbalanced classification, \textit{oversampling}, \textit{small disjuncts}