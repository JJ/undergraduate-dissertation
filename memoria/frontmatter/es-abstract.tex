El siguiente trabajo contiene una formalización matemática del aprendizaje automático, centrándonos en clasificación binaria,
conocida como aprendizaje PAC. Para su formalización ofreceremos una introducción a $\sigma$-álgebras y $\sigma$-álgebras producto y demostraremos desigualdades
clave para construir nuestra teoría PAC. Nuestro resultado más importante será el teorema fundamental del
aprendizaje PAC, que relaciona estadística, combinatoria y aprendizaje automático. También describiremos otro paradigma de aprendizaje,
más laxo que la cognoscibilidad PAC: el aprendizaje no-uniforme.

El desarrollo informático presenta el problema de la clasificación no balanceada. Describiremos distintas aproximaciones 
a la resolución del problema, centrándonos en una, el \textit{oversampling}. Dentro del \textit{oversampling} reseñaremos
las ideas subyacentes en una serie de algoritmos recientes, aún no disponibles en \R, y los implementaremos en dicho 
lenguaje, apoyándonos en \texttt{C++} para acelerar algunos de ellos. Por último, se hará una pequeña experimentación
con \textit{small disjuncts}, un concepto estrechamente ligado al debalanceo de clases.

\paragraph{Palabras clave}
Aprendizaje automático, aprendizaje PAC, clasificación no balanceada, \textit{oversampling}, \textit{small disjuncts}