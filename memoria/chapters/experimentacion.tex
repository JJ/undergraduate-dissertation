\label{ch:sdisjuncts}
En este capítulo proporcionamos una sencilla experimentación de los algoritmos implementados, aplicándolos al concepto de 
\textit{small disjuncts} que se mencionó en \ref{sec:desbalanceo}, problema que surge como consecuencia del desbalanceo
intra clases.

\section{Marco de experimentación}
Se estudiarán los algoritmos MWMOTE, RWO, PDFOS sobre todos los datasets incluidos en el paquete \texttt{imbalance}. 
También se aplicará el algoritmo wRACOG sobre el único dataset de los del paquete enteramente discreto: \texttt{wisconsin},
puesto que en el resto de casos, con atributos continuos, no tendría sentido aplicar el algoritmo. También se hará una 
segunda experimentación, añadiendo filtrado con NEATER a cada uno de los algoritmos y datasets a los que pueden aplicarse.

Trabajaremos con la implementación de árboles C4.5 de \citepalias{rweka}: \texttt{J48}, usando árboles sin poda 
(opción \texttt{-U}), y sin mínimo de ejemplos por hoja (opción \texttt{-M=1}). 
Definiremos un \textit{small disjunct} como una hoja del árbol con 3 o menos ejemplos. Nótese 
que no hay definición formal para el concepto, pues este se define como reglas que cubren pocos ejemplos. Calcularemos para 
cada dataset el número de \textit{small disjuncts} (de ambas clases, puesto que se explicó en \ref{sec:desbalanceo} que 
es un problema que puede afectar a ambas clases), así como la media del número de ejemplos que cubren las hojas del árbol.
Si alguno de los algoritmos de \textit{oversampling} introdujera más \textit{small disjuncts} en lugar de eliminarlos, 
la media de cobertura de las hojas bajaría, y el número de \textit{small disjuncts} crecería. Por tanto teorizamos que si eliminamos 
desbalanceo en los datasets, el número de \textit{small disjuncts} decrecerá, y la media de ejemplos por hoja aumentará.

Se dividirá cada conjunto en 3 particiones estratificadas por clase (esto es, la proporción de ejemplos de ambas
clases será aproximadamente igual que en el conjunto sin particionar). Se hará una media de las medidas descritas para cada
dataset, y compararemos los resultados con los obtenidos para las 3 particiones sin ningún algoritmo aplicado.

La experimentación es completamente reproducible, y se encuentra disponible en la carpeta \texttt{experimentacion} del 
repositorio que aloja al presente trabajo \footnote{\url{https://github.com/ncordon/tfg}}.
Ahí pueden encontrarse dos archivos: \texttt{aux.R} y \texttt{small-disjuncts.R}. Basta ejecutar el segundo archivo
para reproducir la experimentación.

\section{Contenido de \texttt{aux.R}}
\texttt{RWeka} no proporciona ningún método para obtener la cobertura de las hojas de los árboles que construye, así que
ha sido necesario un método para \textit{parsear} la impresión que se hace del árbol en forma de cadena de caracteres

\lstinputlisting[caption=función \texttt{leavesCoverage} en \texttt{aux.R}]{./codelst/leavesCoverage.R}

Asimismo, en dicho archivo también podemos encontrar las funciones \texttt{makePartition}, que recibe un \texttt{dataset}
y un número de particiones \texttt{numPartitions} y genera \texttt{numPartitions} estratificadas a partir del dataset; la
función \texttt{infoSmallDisjuncts} que recibe un \texttt{dataset} y devuelve el cálculo de las dos medidas descritas, y 
la función \texttt{getResults}, que recibe un parámetro \textit{booleano} indicando si queremos aplicar el filtro NEATER
o no.
\clearpage
\lstinputlisting[caption=función \texttt{infoSmallDisjuncts} en \texttt{aux.R}]{./codelst/infoSmallDisjuncts.R}

\section{Contenido de \texttt{small-disjuncts.R}}
Este archivo es el fichero principal de la experimentación. Comienza fijando una semilla aleatoria de modo que la experimentación
sea reproducible, carga librerías, el fichero auxiliar \texttt{aux.R}, los datos de \texttt{imbalance}, los particiona,
crea un \textit{wrapper} para \texttt{wRACOG} basado en \texttt{J48} y obtiene los resultados de la experimentación.

\section{Resultados}
En los resultados, \texttt{none} representa el dataset original sin tratamiento de \textit{oversampling}.
\subsection{Sin filtrado de instancias}
  \begin{table}[H]
  \centering
  \begin{tabular}{rrrrrr}
    \hline
    & none & mwmote & wracog & rwo & pdfos \\ 
    \hline
    ecoli1 & 24.83 & 17.88 &  & 11.22 &  \\ 
    glass0 & 12.73 & 11.90 &  & 15.71 &  \\ 
    haberman & 16.19 & 15.02 &  & 24.72 & 21.77 \\ 
    iris0 & 25.00 & 30.33 &  & 30.33 & 30.33 \\ 
    newthyroid1 & 16.72 & 30.00 &  & 24.00 & 18.90 \\ 
    wisconsin & 5.79 & 3.54 & 6.22 & 3.74 & 3.74 \\ 
    yeast4 & 62.49 & 25.13 &  & 36.81 &  \\ 
    \hline
  \end{tabular}
  \caption{Media de \textit{small disjuncts} tras \textit{oversampling}}
  \end{table}

  \begin{table}[H]
  \centering
  \begin{tabular}{rrrrrr}
    \hline
    & none & mwmote & wracog & rwo & pdfos \\ 
    \hline
    ecoli1 & 2.67 & 4.33 &  & 18.00 &  \\ 
    glass0 & 3.67 & 5.33 &  & 3.33 &  \\ 
    haberman & 3.33 & 1.67 &  & 2.67 & 2.33 \\ 
    iris0 & 0.00 & 0.00 &  & 0.00 & 0.00 \\ 
    newthyroid1 & 1.67 & 1.33 &  & 1.33 & 2.33 \\ 
    wisconsin & 29.33 & 66.00 & 30.67 & 61.67 & 61.67 \\ 
    yeast4 & 4.33 & 16.00 &  & 26.33 &  \\ 
    \hline
  \end{tabular}
  \caption{Tamaño medio de coberturas tras \textit{oversampling}}
  \end{table}
  
\subsection{Con filtrado de instancias}
  \begin{table}[H]
  \centering
  \begin{tabular}{rrrrrr}
  \hline
  & none & mwmote & wracog & rwo & pdfos \\ 
  \hline
  ecoli1 & 24.83 & 31.36 &  & 16.29 &  \\ 
  glass0 & 12.73 & 14.39 &  & 16.53 &  \\ 
  haberman & 16.19 & 10.17 &  & 26.48 & 10.05 \\ 
  iris0 & 25.00 & 30.33 &  & 30.33 & 30.33 \\ 
  newthyroid1 & 16.72 & 29.06 &  & 21.58 & 22.97 \\ 
  wisconsin & 5.79 & 3.50 & 5.67 & 3.79 & 3.79 \\ 
  yeast4 & 62.49 & 23.66 &  & 28.53 &  \\ 
   \hline
\end{tabular}
  \caption{Media de \textit{small disjuncts} tras \textit{oversampling} y filtrado}
  \end{table}

  \begin{table}[H]
  \centering
  \begin{tabular}{rrrrrr}
    \hline
    & none & mwmote & wracog & rwo & pdfos \\ 
    \hline
    ecoli1 & 2.67 & 1.33 &  & 16.00 &  \\ 
    glass0 & 3.67 & 4.67 &  & 2.67 &  \\ 
    haberman & 3.33 & 5.00 &  & 2.33 & 5.33 \\ 
    iris0 & 0.00 & 0.00 &  & 0.00 & 0.00 \\ 
    newthyroid1 & 1.67 & 1.33 &  & 1.00 & 2.00 \\ 
    wisconsin & 29.33 & 66.33 & 33.00 & 60.33 & 60.67 \\ 
    yeast4 & 4.33 & 19.33 &  & 27.67 &  \\ 
    \hline
  \end{tabular}
  \caption{Tamaño medio de coberturas tras \textit{oversampling} y filtrado}
  \end{table}
  
