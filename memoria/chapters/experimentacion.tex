\label{ch:sdisjuncts}
En este capítulo proporcionamos una sencilla experimentación de los algoritmos implementados, aplicándolos al concepto de 
\textit{small disjuncts} que se mencionó en \ref{sec:desbalanceo}, problema que surge como consecuencia del desbalanceo
\textit{intra clases}.

\section{Marco de experimentación}
Se estudiarán los algoritmos MWMOTE, RWO, PDFOS sobre todos los datasets incluidos en el paquete \texttt{imbalance}. 
También se aplicará el algoritmo wRACOG sobre el único dataset de los del paquete enteramente discreto: \texttt{wisconsin},
puesto que en el resto de casos, con atributos continuos, no tendría sentido aplicar el algoritmo. También se hará una 
segunda experimentación, añadiendo filtrado con NEATER a cada uno de los algoritmos y datasets a los que pueden aplicarse.

Trabajaremos con la implementación de árboles C4.5 de \citepalias{rweka}: \texttt{J48}, usando árboles sin poda 
(opción \texttt{-U}), y sin mínimo de ejemplos por hoja (opción \texttt{-M=1}). 
Definiremos un \textit{small disjunct} como una hoja del árbol con 3 o menos ejemplos. Nótese 
que no hay definición formal para el concepto, pues este se define como reglas que cubren pocos ejemplos. Calcularemos para 
cada dataset el número de \textit{small disjuncts} (de ambas clases, puesto que se explicó en \ref{sec:desbalanceo} que 
es un problema que puede afectar a ambas clases), así como la media del número de ejemplos que cubren las hojas del árbol.

Si alguno de los algoritmos de \textit{oversampling} introdujera más \textit{small disjuncts} en lugar de eliminarlos, 
la media de cobertura de las hojas bajaría, y el número de \textit{small disjuncts} crecería. Por tanto teorizamos que si eliminamos 
desbalanceo en los datasets, el número de \textit{small disjuncts} decrecerá, y la media de ejemplos por hoja aumentará. 
También trabajaremos con la sensibilidad, medida que se explicó en la sección \ref{sec:medidas-desbalanceo}.

Se dividirá cada conjunto en 3 particiones estratificadas por clase (esto es, la proporción de ejemplos de ambas
clases será aproximadamente igual que en el conjunto sin particionar), y se intentará conseguir un ratio de desbalanceo del
80\% con cada uno de los algoritmos. Se hará una media de las medidas descritas para cada dataset, y compararemos los 
resultados con los obtenidos para las 3 particiones sin ningún algoritmo aplicado. Para aplicar el algoritmo wRACOG se usará
como conjunto de entrenamiento una partición y como conjunto de validación la siguiente.

La experimentación es completamente reproducible, y se encuentra disponible en la carpeta \texttt{experimentacion} del 
repositorio que aloja al presente trabajo \footnote{\url{https://github.com/ncordon/tfg}}.
Ahí pueden encontrarse dos archivos: \texttt{aux.R} y \texttt{small-disjuncts.R}. Basta ejecutar el segundo archivo
para reproducir la experimentación.

\section{Contenido de \texttt{aux.R}}
\texttt{RWeka} no proporciona ningún método para obtener la cobertura de las hojas de los árboles que construye, así que
ha sido necesario un método para \textit{parsear} la impresión que se hace del árbol en forma de cadena de caracteres
\lstinputlisting[caption=función \texttt{leavesCoverage} en \texttt{aux.R}]{./codelst/leavesCoverage.R}

Asimismo, en dicho archivo también podemos encontrar las funciones \texttt{makePartition}, que recibe un \texttt{dataset}
y un número de particiones \texttt{numPartitions} y genera particiones estratificadas a partir del dataset; la
función \texttt{infoSmallDisjuncts} que recibe un \texttt{dataset} y devuelve el cálculo de las dos medidas descritas, y 
la función \texttt{getResults}, que recibe un parámetro \textit{booleano} indicando si queremos aplicar el filtro NEATER
o no. Las funciones para particionar y obtener resultados fijan la semilla $12345$ por defecto, con objeto de que la 
experimentación sea reproducible.
\lstinputlisting[caption=función \texttt{infoSmallDisjuncts} en \texttt{aux.R}]{./codelst/infoSmallDisjuncts.R}

\section{Contenido de \texttt{small-disjuncts.R}}
Este archivo es el fichero principal de la experimentación. Carga librerías, el fichero auxiliar \texttt{aux.R}, los datos
de \texttt{imbalance}, los particiona, crea un \textit{wrapper} para \texttt{wRACOG} basado en \texttt{J48} y obtiene los 
resultados de la experimentación.

\section{Resultados}
En los resultados, \texttt{none} representa el dataset original sin tratamiento de \textit{oversampling}. En la columna
correspondiente al algoritmo PDFOS faltan algunos datos debido a que las matrices de covarianza de los datos particionados 
resultan no invertibles, tal y como se explicó en la sección \ref{sec:critica-pdfos}. La columna de wRACOG sólo incluye el
único dataset discreto, \texttt{wisconsin}.
\subsection{Sin filtrado de instancias}
  \begin{table}[H]
  \centering
  \begin{tabular}{rrrrrr}
  \hline
  & none & mwmote & wracog & rwo & pdfos \\ 
  \hline
  ecoli1 & 2.67 & 3.00 &  & 17.67 &  \\ 
  glass0 & 2.33 & 5.00 &  & 3.67 &  \\ 
  haberman & 1.33 & 6.33 &  & 3.33 & 4.33 \\ 
  iris0 & 0.00 & 0.00 &  & 0.00 & 0.00 \\ 
  newthyroid1 & 1.67 & 1.33 &  & 1.67 & 1.00 \\ 
  wisconsin & 27.67 & 62.33 & 36.00 & 65.67 & 65.67 \\ 
  yeast4 & 3.00 & 16.67 &  & 5.00 &  \\ 
  \hline
  \end{tabular}
  \caption{Media de \textit{small disjuncts} tras \textit{oversampling}}
  \end{table}

  \begin{table}[H]
  \centering
  \begin{tabular}{rrrrrr}
  \hline
  & none & mwmote & wracog & rwo & pdfos \\ 
  \hline
  ecoli1 & 24.83 & 20.39 &  & 14.13 &  \\ 
  glass0 & 13.30 & 8.76 &  & 11.07 &  \\ 
  haberman & 45.05 & 14.86 &  & 20.00 & 22.31 \\ 
  iris0 & 25.00 & 30.33 &  & 30.33 & 30.33 \\ 
  newthyroid1 & 19.89 & 30.00 &  & 20.40 & 23.40 \\ 
  wisconsin & 6.74 & 3.88 & 5.68 & 3.65 & 3.65 \\ 
  yeast4 & 77.83 & 30.69 &  & 55.21 &  \\ 
  \hline
  \end{tabular}
  \caption{Tamaño medio de coberturas tras \textit{oversampling}}
  \end{table}
  
  \begin{table}[H]
  \centering
  \begin{tabular}{rrrrrr}
  \hline
  & none & mwmote & wracog & rwo & pdfos \\ 
  \hline
  ecoli1 & 0.76 & 1.00 &  & 0.96 &  \\ 
  glass0 & 0.97 & 0.98 &  & 1.00 &  \\ 
  haberman & 0.54 & 0.85 &  & 0.76 & 0.68 \\ 
  iris0 & 1.00 & 1.00 &  & 1.00 & 1.00 \\ 
  newthyroid1 & 0.97 & 1.00 &  & 0.99 & 1.00 \\ 
  wisconsin & 1.00 & 1.00 & 1.00 & 1.00 & 1.00 \\ 
  yeast4 & 0.33 & 1.00 &  & 0.99 &  \\ 
  \hline
  \end{tabular}
  \caption{Media de sensibilidad tras \textit{oversampling}}
  \end{table}
  
  
\subsection{Con filtrado de instancias}
  \begin{table}[H]
  \centering
  \begin{tabular}{rrrrrr}
  \hline
  & none & mwmote & wracog & rwo & pdfos \\ 
  \hline
  ecoli1 & 2.67 & 2.67 &  & 17.67 &  \\ 
  glass0 & 2.33 & 3.67 &  & 3.33 &  \\ 
  haberman & 1.33 & 7.33 &  & 3.00 & 5.33 \\ 
  iris0 & 0.00 & 0.00 &  & 0.00 & 0.00 \\ 
  newthyroid1 & 1.67 & 1.33 &  & 2.33 & 1.00 \\ 
  wisconsin & 27.67 & 62.33 & 34.33 & 65.00 & 63.67 \\ 
  yeast4 & 3.00 & 9.67 &  & 4.00 &  \\ 
  \hline
  \end{tabular}
  \caption{Media de \textit{small disjuncts} tras \textit{oversampling} y filtrado}
  \end{table}

  \begin{table}[H]
  \centering
  \begin{tabular}{rrrrrr}
   \hline
   & none & mwmote & wracog & rwo & pdfos \\ 
   \hline
  ecoli1 & 24.83 & 19.59 &  & 14.10 &  \\ 
  glass0 & 13.30 & 9.80 &  & 14.77 &  \\ 
  haberman & 45.05 & 9.37 &  & 14.32 & 12.55 \\ 
  iris0 & 25.00 & 30.33 &  & 30.33 & 30.33 \\ 
  newthyroid1 & 19.89 & 28.83 &  & 21.19 & 23.85 \\ 
  wisconsin & 6.74 & 3.88 & 5.78 & 3.67 & 3.71 \\ 
  yeast4 & 77.83 & 35.66 &  & 52.80 &  \\ 
   \hline
  \end{tabular}
  \caption{Tamaño medio de coberturas tras \textit{oversampling} y filtrado}
  \end{table}
  
  \begin{table}[H]
  \centering
  \begin{tabular}{rrrrrr}
  \hline
  & none & mwmote & wracog & rwo & pdfos \\ 
  \hline
  ecoli1 & 0.76 & 0.98 &  & 0.96 &  \\ 
  glass0 & 0.97 & 0.98 &  & 0.99 &  \\ 
  haberman & 0.54 & 0.83 &  & 0.85 & 0.87 \\ 
  iris0 & 1.00 & 1.00 &  & 1.00 & 1.00 \\ 
  newthyroid1 & 0.97 & 1.00 &  & 0.99 & 1.00 \\ 
  wisconsin & 1.00 & 1.00 & 1.00 & 1.00 & 1.00 \\ 
  yeast4 & 0.33 & 0.99 &  & 0.99 &  \\ 
  \hline
  \end{tabular}
  \caption{Media de sensibilidad tras \textit{oversampling} y filtrado}
  \end{table}
  
  Se puede observar cómo nuestros algoritmos consiguen en algunos casos reducir el número de hojas de pequeña 
  cobertura existente en los datasets, y aumentar los tamaños medios de las mismas. En especial se desprende que el algoritmo
  NEATER mejora los resultados obtenidos sin filtrado. Hemos usado en todos ellos los parámetros por defecto, y podrían 
  obtenerse mejores resutlados si intentásemos optimizar los mismos. Además, la medida de \textit{small disjuncts} que está
  siendo empleada es bastante arbitraria, y podría intentarse encontrar una más adaptada a cada dataset: hojas que tuviesen
  una cobertura inferior al 80\% de la media, por ejemplo. 
  
  A esto hay que sumar que en algunos casos \texttt{J48} puede estar creando una única hoja donde clasifique a todos los
  ejemplos dentro  de la misma clase, con lo cual obtendríamos un tamaño de cobertura bastante grande. Ello no quiere decir
  que si el \textit{oversampling} disminuye el tamaño medio de cobertura, su funcionamiento no sea el correcto. Prueba de esto
  es que los algoritmos implementados aumentan notablemente la sensibilidad de los modelos, lo cual indica que 
  consiguen su objetivo de mejorar la clasificación de instancias positivas.
 