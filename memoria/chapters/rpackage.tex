En este capítulo presentamos el paquete de \R desarrollado: \texttt{imbalance}, así como las tecnologías y 
metodologías empleadas en su desarrollo.

\section{¿Qué es R? ¿Por qué R?}
\citepalias{rlang} es un lenguaje pensado para estadística computacional y todo lo que ello implica: 
manipulación de datos, visualización, variedad de modelos estadísticos, etc. \R puede ser considerado una 
implementación renovada del antiguo lenguaje \texttt{S}, para estadística. Destaca sobretodo por su amplia gama de
paquetes, disponible en repositorios como CRAN \footnote{\url{https://cran.r-project.org/}} o 
Bioconductor \footnote{\url{https://www.bioconductor.org}}, este último orientado a bioinformática.

\R sobresale por su extensibilidad, su facilidad de uso, su extensa documentación (para que un paquete pueda ser
publicado en CRAN tiene como requisito fundamental estar bien documentado), su fácil acceso a ella,
y su heterogénea comunidad de usuarios, siendo empleado por estadistas, bioinformáticos, científicos
de datos, informáticos o incluso investigadores asociados a medicina o biología.

\R, a diferencia de otros lenguajes, no tiene un estándar y una implementación independientes, puesto que lo que se
asume como estándar del lenguaje es la propia implementación GNU-R (implementación libre), a pesar de existir otras 
como \texttt{pqR} \footnote{\url{www.pqr-project.org}} (\textit{Pretty Quick R}) o 
\texttt{FastR} \footnote{\url{https://github.com/allr/purdue-fastr}}. \R no es rápido, puesto que es un 
lenguaje interpretado y el \textit{core} de GNU-R no tiene como objetivo fundamental hacerlo más rápido, sino
proporcionar un lenguaje estable. 

La filosofía principal de la programación en R es:

\begin{itemize}
  \item \textbf{Programación funcional}: los paquetes desarrollados sólo pueden exportar funciones (técnicamente
  también ojetos \texttt{S3}, \texttt{S4} o \texttt{RC}, pero incluso los mismos trabajan con un tipo de orientación
  a objetos peculiar, donde la función recibe al objeto, y este sólo se usa para resolver sobre qué método concreto
  de los que concretan la función se \textit{despacha} el objeto).
  \item \textbf{Inmutabilidad}: cualquier función del lenguaje no puede modificar un objeto que existe en
  un entorno o capa superior desde donde ha sido llamada la función (técnicamente puede hacerlo, pero con metaprogramación,
  no con evaluación convencional).
  \item \textbf{Vectorización}: \R incluye una serie de funcionales (funciones que toman otras como entrada) 
  que vectorizan operaciones (hacen que se ejecuten de manera más eficiente, aprovechando la estructura del objeto)
  sobre matrices, vectores, datasets, listas, etc. Cuando puede usarse vectorización, es mucho más
  rápida que los bucles convencionales.
\end{itemize}

Se ha escogido \R como herramienta para desarrollar el software dada su arraigado uso en estadística y en ciencia
de datos. Aunque otros lenguajes como \texttt{Python} proporcionan herramientas tremendamente potentes para
hacer ciencia de datos, \R parece estar menos orientado a programadores puros y más a obtener resultados
científicos a partir de lo programado. Se quería proporcionar una implementación de los algoritmos que 
ofreciera un balance entre eficiencia y facilidad de uso, y \R parecía la mejor opción para alcanzar dicho
objetivo. Además, la diseminación de los algoritmos principales (no únicamente en preprocesamiento de datos,
sino en ciencia de datos en general) en varios paquetes (entre los de tratatamiento de datos no balanceados cabe
citar a \citepalias{rsmote} o \citepalias{rrose} en \R) hace que no haya una fuerte dependencia de un 
único paquete de software, como ocurriría con \citepalias{scikit} en \texttt{python}.

\section{Instalación}
\subsection{Instalación de \R}
La instalación de \R en distribuciones Linux basadas en \texttt{Debian} puede efectuarse de la forma:
  \begin{lstlisting}[language=R,numbers=none]
  sudo apt install r-base rbase-dev
  \end{lstlisting}

En distribuciones basadas en \texttt{Arch Linux}:
  \begin{lstlisting}[language=R,numbers=none]
  sudo pacman -S r
  \end{lstlisting}

También es recomendable, aunque no necesario, instalar \texttt{RStudio} \footnote{\url{https://www.rstudio.com}}.

\subsection{Instalación del software}
La instalación del paquete \texttt{imbalance} desarrollado se puede llevar a cabo una vez abierto desde 
una terminal \R o \texttt{RStudio} usando el paquete \citepalias{rdevtools}:
  \begin{lstlisting}[language=R,numbers=none]
  install.packages("devtools")
  devtools::install_github("ncordon/imbalance")
  \end{lstlisting}

Posterior a este paso, sólo falta cargar el paquete:

  \begin{lstlisting}[language=R,numbers=none]
  library(imbalance)
  \end{lstlisting}

La instalación del paquete \texttt{devtools} (primera línea) no es necesaria si ya se tiene instalado.

Cualquier cuestión de documentación puede ser consultada en el anexo del presente trabajo, en la página
de documentación online \url{https://ncordon.github.io/imbalance}, pestaña \textit{Reference}; o usando la 
ayuda de manual de \R, una vez cargado el paquete, con el nombre de la función precedido de \texttt{?} (ej. \texttt{?rwo}).

\imgcaption{./imgs/docweb.png}{Web de documentación del proyecto}{0.95}

Los métodos que exporta el paquete y que pueden usarse una vez
se carga son:

\begin{itemize}
 \item \texttt{mwmote}
 \item \texttt{racog}
 \item \texttt{wracog}
 \item \texttt{rwo}
 \item \texttt{pdfos}
 \item \texttt{neater}
 \item \texttt{plotComparison}
\end{itemize}

\section{Estructura del paquete}
El paquete, cuyo código fuente está disponible en un repositorio de Github \footnote{\url{https://github.com/ncordon/imbalance}}, 
bajo licencia GPL2 y posteriores, presenta la siguiente estructura de directorios y archivos:

\begin{itemize}
 \item \R: En esta carpeta se encuentra código fuente de los algoritmos, cada uno en un archivo homónimo al
 nombre del algoritmo. También se encuentran los siguientes archivos:
 
 \begin{itemize}
  \item \texttt{data.R}: contiene documentación en forma de comentarios de \texttt{Roxygen2} de la estructura
    de los datasets incluidos con el paquete, todos ellos extraídos del repositorio \citepalias{keel}.
  \item \texttt{RccpExports.R}: usado por la librería \texttt{Rcpp} para almacenar wrappers de llamadas a funciones
    de \texttt{C++} que se encuentran en la carpeta \texttt{src} de la raíz del paquete.
  \item \texttt{zzz.R}: (nombre del archivo por convenio), que se usa para liberar las librerías dinámicas generadas con 
    \texttt{C++} en caso de que se libere el \texttt{Namespace} asociado al paquete.
  \item \texttt{imbalance.R}: contiene la documentación del paquete para la ayuda \texttt{help}, accesible a través de:
   \begin{lstlisting}[language=R, numbers=none]
   help(package="imbalance")
   \end{lstlisting}
  \item \texttt{utils.R}: contiene funciones auxiliares llamadas desde los algoritmos, y la función \texttt{plotComparison}
    para evaluar visualmente los resultados de un \textit{oversampling} o \textit{undersampling}.
 \end{itemize}
 
 \item \texttt{data-raw}: hay un script \texttt{rkeel-data.R} en esta carpeta que contiene el procedimiento con el que se han
 leído y limpiado los datos desde archivos en formato KEEL. Es habitual distribuir este tipo de archivos en los fuentes de
 los paquetes, aunque las \textit{builds} de los mismos no los contienen, por estar añadida esta carpeta al archivo \texttt{.Rbuildignore}.
 \item \texttt{data}: contiene archivos de datos \texttt{.rda} con los datasets con el tipo de datos 
 correcto para cada columna y con el atributo de clase ``Class'' con posibles valores \textit{positive} o \textit{negative}, 
 respetando su formato original. Ha sido necesario consultar no sólo el repositorio KEEL, sino también el repositorio 
 de \textit{machine learning} de \citepalias{uci}. Para acceder a la documentación sobre el formato de cada uno de ellos 
 basta escribir en la consola de \R el nombre del dataset precedido de \texttt{?} (ej. \texttt{?newthryoid1}). Los
 siguientes datasets están disponibles: \texttt{ecoli1}, \texttt{glass0}, \texttt{haberman}, \texttt{iris0},
 \texttt{newthryoid1}, \texttt{wisconsin}, \texttt{yeast4}.
 \item \texttt{src}: contiene código auxiliar en \texttt{C++} usado para acelerar el cuerpo de los algoritmos homónimos. El
 archivo \texttt{RcppExports.cpp} contiene cabeceras de funciones que genera automáticamente el paquete \texttt{Rcpp}.
 \item \texttt{tests}: contiene los tests que pasa el paquete en la integración continua. Cada vez que una modificación 
 es subida al repo se comprueba que ninguna función de las exportadas por el paquete falle, que las funciones comprueben
 correctamente los parámetros de entrada, que el paquete pase los tests en formato CRAN, etc.
\end{itemize}


\section{Metodología de desarrollo}
Los algoritmos han sido fundamentalmente programados en \R, de manera que todos tienen una estructura de la forma:
\lstinputlisting[language=R, numbers=none]{./codelst/algorithm.R}

Asimismo, se ha usado la librería \citepalias{rcpp}, que proporciona una API de integración con \texttt{C++}, de manera que
no hay que emplear manualmente llamadas a \texttt{.Call} en \R y facilita la transformación de tipos entre ambos lenguajes. También 
ofrece azúcar sintáctico inspirado en \R disponible al programar \texttt{C++} apoyándonos en \texttt{Rcpp}. Además, para el desarrollo
de dichos programas auxiliares hemos tenido que usar la librería \citepalias{rarmadillo}, que proporciona una extensa variedad 
de operaciones matriciales y vectoriales en \texttt{C++}, y cuya sintaxis está inspirada en \texttt{Octave}
\footnote{\url{https://www.gnu.org/software/octave}}.

Por último, cabe comentar que se ha empleado \texttt{Travis CI} \footnote{\url{https://travis-ci.org}} como servicio de integración
continua con el repo de Github. De esta forma se han automatizado ejecuciones de tests unitarios y generación de documentación web. El
archivo de configuración empleado para el servicio de integración continua está disponible en la raíz del repo, con nombre 
\texttt{.travis.yml}.

\section{Ejemplo de uso del software}
Ejecución del algoritmo PDFOS sobre el dataset \texttt{newthryoid1} y evaluación de resultados visualmente con la función
\texttt{plotComparison}:

\lstinputlisting{./codelst/ex-pdfos.R}
\imgcaption{../imbalance/README-example-pdfos-1.png}{
  PDFOS sobre los 3 primeros atributos de \texttt{newthryoid1}}{1.1}

Limpiado de las instancias generadas mediante NEATER:

\lstinputlisting{./codelst/ex-neater.R}
\imgcaption{../imbalance/README-example-neater-1.png}{
  Evaluación de NEATER sobre las instancias anteriores}{1.1}
