\section{Algoritmo PDFOS}

\subsection{Motivación}
Dada una función de distribución de $X$ variable aleatoria, $F(x)$, si esta función es derivable casi seguramente entonces
podemos tomar la función de densidad como la derivada de la función de distribución, casi seguramente, es decir:
\[f(x) = \lim_{h\rightarrow 0} \frac{F(x+h) - F(x-h)}{2h} = \lim_{h\rightarrow 0} \frac{P(x-h < X \le x+h)}{2h}\]

Si tenemos muestras aleatorias de $X$, a saber, $X_1, \ldots X_n$ y $x_1, \ldots x_n$ una realización muestral, entonces un 
estimador para $f$ sería:
\[\widehat{f}(x) = \frac{1}{2hn} \bigg[\textrm{Número de } x_1, \ldots, x_n \textrm{ que se quedan en ]x-h, x+h[}\bigg]\]

Es decir, definiendo $\omega(x) = \left\{\begin{array}{ll} 
                                \frac{1}{2} &, |x| < 1\\
                                0 & \textrm{en otro caso}
                                \end{array}\right.$
                                
                                
y $w_h(x) = w\left(\left|\frac{x}{h}\right|\right)$, podríamos reformular $\widehat{f}$ como:
\[\widehat{f}(x) = \frac{1}{nh} \sum_{i=1}^n \omega_h(x-x_i)\]

Es decir, supuestas que las observaciones $x_1, \ldots, x_n$ se distancian múltiplos de $2h$ (caen en el centro de intervalos
de longitud $2h$, habríamos construido $\widehat{f}$ a base de un histograma donde cada barra tiene ancho $2h$ y altura 
$\frac{1}{2nh} \cdot \bigg[[\textrm{Número de muestras } x_1, \ldots, x_n \textrm{ en el intervalo}]\bigg]$. Al parámetro $h$
se le llama parámetro de \textit{bandwidth}, por referencia al significado que tiene en el caso de los histogramas.

\subsection{Generalización a funciones kernel}
Tomar $w = \frac{1}{2} \mathds{1}_{]-1,1[}$ presenta el problema de que $\widehat{f}$ será una función a saltos, y no
será continua. Surge una generalización al tomar $\omega$ como funciones verificando $\int_{\Omega} \omega(x) dx = 1$, 
siendo $\Omega$ el dominio de $X$. Se suelen considerar además funciones simétricas $w(x) = w(-x)$ para cualquier 
$x\in \Omega$.

El estimador que se suele usar para evaluar la bondad de $\widehat{f}$ es el error cuadrático medio integrado (MISE):

\[MISE(h) = \underset{x_1, \ldots, x_n}{\expect} \int (\widehat{f}(x) - f(x))^2 dx\]

\subsection{Funciones kernel gaussianas multivariantes}
El algoritmo PDFOS (\textit{Probability Distribution Oversampling}) se basa en tomar funciones kernel gaussianas multivariantes.
Recordamos la definición de la función de densidad de la distribución Gaussiana multivariante de media $0$ y covarianza $\Psi$:

\[\phi^{\Psi}(x) = \frac{1}{\sqrt{2\pi \cdot det(\Psi)}} exp\left(\frac{1}{2} x \Psi^{-1} x^T \right)\]

Dada $\spos = \{x_1, \ldots, x_m\}$, la clase minoritaria, calcularemos el estimador no sesgado para la covarianza:

\[U = \frac{1}{m-1} \sum_{i=1}^m (x_i - \overline{x})(x_i - \overline{x})^T, 
  \qquad \textrm{siendo } \overline{x} = \frac{1}{M}\sum_{i=1}^M x_i\]
  
Las funciones kernel que tomaremos serán de la forma: $\phi = \phi^{U}$, y tendremos que ajustar el \textit{bandwidth} $h$ de
$\phi_h(x) = \phi^U\left(\frac{x}{h}\right)$ para minimizar el MISE. A tales efectos, hay que minimizar la función de validación
cruzada:
\[M(h) = \frac{1}{m^2} \sum_{i=1}^m \sum_{j=1}^m \phi_h^{\ast} (x_i - x_j) + \frac{2}{m} \phi_h(0)\]

donde se toma la aproximación:
\[\phi_h^{\ast} \approx \phi_{h\sqrt{2}} - 2\phi_h\]

Una vez minimizada la función de validación cruzada $M$, el esquema de generación de instancias se basará en dada una instancia
$x_i \in \spos$, tomar $x_i + h R r$, donde $r\sim N^m(0,1)$ y $R$ es la matriz triangular superior resultante de la descomposición
de Choleski. Análogamente al caso unidimensional, donde se pueden generar instancias desde $N(\mu, \sigma)$ tomando $\mu + \sigma + r$
con $r\sim N(0,1)$, en el caso multivariante dada $\Psi =  A \cdot A^T$

\begin{algorithm}[H]
\begin{algorithmic}[1]
  \REQUIRE $\spos = \{z_1=(x_1, y_1), \ldots z_m=(x_m, y_m)\}$, ejemplos positivos
  \REQUIRE $T$, número de instancias sintéticas deseado
  \STATE{$S = \spos_x = \{x_i=(w_1^{(i)}, \ldots w_d^{(i)})\}_{i=1}^m$}
  \STATE{$S'= \emptyset$}
  \STATE{$\sigma =$ búsqueda con GridSearch que minimice $M$}
  \STATE{Calcular $U$ la matriz de covarianza de $S$}
  \STATE{Calcular descomposición de Choleski de $U$, donde $U=R^{T} R$, y $R$ triangular superior}
  \NEWLINE
  \FOR{$i=1, \ldots, T$}
    \STATE{Escoger $x\in S$}
    \STATE{Escoger $r$ siguiendo una normal multivariante, $r \sim N^d(0,1)$}
    \STATE{$S' = S' \cup \{x + \sigma r R\}$}
  \ENDFOR
  \NEWLINE
  \RETURN{$S'$, ejemplos positivos sintéticos}
\end{algorithmic}
\caption{Algoritmo de \textit{oversampling} PDFOS}
\label{alg:pdfos}
\end{algorithm}


Se comenzó usando el siguiente algoritmo de búsqueda \textit{grid}:

\begin{algorithm}[H]
\begin{algorithmic}[1]
  \STATE{Hacer $M_{best} = \infty$}
  \FOR{$\tau \in \{0.2, 0.22, 0.24, \ldots 2\}$}
    \IF{$M(\tau) < M_{best}$}
      \STATE{$\sigma = \tau, M_{best} = M$}
    \ENDIF
  \ENDFOR
  \RETURN{$\sigma$}
\end{algorithmic}
\caption{Algoritmo de búsqueda GridSearch}
\end{algorithm}

Posteriormente se agregaron también al algoritmo de búsqueda la comprobación con los valores de parámetro propuestos por Scott
y Silverman (siendo $d$ el tamaño del espacio de atributos, y $m$ el número de instancias pertenecientes a la clase minoritaria):

\[\sigma_{Scoot} = \left(\frac{1}{m}\right)^{\frac{1}{d+4}}, \quad \sigma_{Silverman} = \left(\frac{4}{m(d+2)}\right)^{\frac{1}{d+4}}\]
