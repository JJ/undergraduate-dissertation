\section{Algoritmo RWO}
El algoritmo RWO (\textit{Random Walk Oversampling}) consiste en construir instancias sintéticas inspirándonos
en el teorema central del límite, de manera que intentamos dejar invariante el límite en distribución de los
datos de entrenamiento.

\begin{theorem}
 Sean $\{W_1, \ldots, W_m\}$ un conjunto de variables aleatorias i.i.d., con $\expect(W_i) = \mu$ y 
 $Var(W_i) = \sigma^2 < \infty$. Entonces:
 
 \[\lim_{m} P\left[\frac{\sqrt{m}}{\sigma} \left(\underbrace{\frac{1}{m}\sum_{i=1}^m W_i}_{\overline{W}} - 
   \mu \right) \le z \right] = \phi(z)\]
 
 donde $\phi$ es la función de distribución de la normal $N(0,1)$.
 
 Es decir $\frac{\overline{W} - \mu}{\sigma/\sqrt{m}} \rightarrow N(0,1)$ en probabilidad.
 
 \label{th:tcl}
\end{theorem}

Supongamos que las muestras son de dimensión $d$. Fijamos una columna $j\in \{1, \ldots d\}$. Supuesto un
conjunto de instancias de la clase positiva $\{x_i=(w_1^{(i)}, \ldots w_d^{(i)})\}_{i=1}^m$, supondremos que la
columna $j-$ésima de los datos está generada según una variable aleatoria $W_j$ de media $\mu_j$ y desviación
típica $\sigma_j < \infty$.

Sean $\mu_j', \sigma_j'$ la media y desviación típica muestrales, respectivamente. Las instancias sintéticas 
que generaremos serán de la forma $w_j^{'(i)} = w_j^{(i)} - \frac{\sigma_j'}{\sqrt{m}} \cdot r_i$, con 
$r_i\sim N(0,1)$, para $i=1, 2, \ldots, m$. Nótese la similitud de esta fórmula con \ref{th:tcl}, con la salvedad
de que usamos los valores para la variable $W_j$ que conocemos en lugar de $\mu$, que es desconocida; y 
$\sigma_j'$ que es la varianza muestral no corregida, en lugar de la verdadera varianza muestral $\sigma_j$, 
que tampoco la conocemos.

\begin{theorem}
 La esperanza de la media muestral de las instancias $\{w_j^{'(i)}\}_{i=1}^m$ sintéticas es $\mu_j$. 
 La esperanza de la varianza muestral tiende en $m$ a $\sigma_j^2$.
\end{theorem}

\begin{proof}
Para la media:
 
 \[\expect(W_j') = \expect(W_j) - \frac{\sigma_j'}{\sqrt{m}} \expect(r_i) = \mu_j\]
 
 Puesto que $\expect(W_j) = \mu$ y $\expect(r_i) = 0$. Usando linealidad de la esperanza se prueba que
 $\expect\left(\frac{1}{m} \sum_{i=1}^m w_j'(i)\right) = \mu_j$.
 
 Ahora dado $j$ arbitrario:
 
 \begin{align*}
   \expect \bigg(\bigg(W_j - \frac{r_i}{\sqrt{m}} \sigma_j' - \mu_j\bigg)^2\bigg) &= 
   \expect \bigg((W_j - \mu_j)^2 \bigg) - \expect\bigg( 2 (W_j-\mu_j) \frac{r_i}{\sqrt{m}} \sigma_j' \bigg) + \\
   &+ \expect\left(\frac{r_i^2}{m} \sigma_j'^2 \right) = \sigma_j^2 + 0 + \frac{1}{m}\sigma_j'^2
 \end{align*}
 
 Sea $\tau_m = \frac{1}{m} \left(\sum_{i=1}^m w_j'(i) - \frac{1}{m} \sum_{i=1}^m w_j'(i)\right)^2$ la varianza muestral.
 
 Es conocido que $\expect(\tau) = \frac{m-1}{m} Var(W_j') = 
 \frac{(m-1)}{m} \left(\sigma_j^2 + \frac{1}{m} \sigma_j'^2\right)$

 Luego $\lim_{m\rightarrow + \infty} \expect(\tau_m) = \sigma_j$.
\end{proof}

El algoritmo incluye una distinción para los atributos no numéricos, donde se establece una distribución uniforme
para todos los valores que posean dichos atributos, y se le asigna a la instancia generada. En pseudocódigo
obtenemos:

\begin{algorithm}[H]
\begin{algorithmic}[1]
  \REQUIRE $\spos = \{z_1=(x_1, y_1), \ldots z_m=(x_m, y_m)\}$, ejemplos positivos
  \REQUIRE $T$, número de instancias sintéticas deseado
  \STATE{$S = \spos_x = \{x_i=(w_1^{(i)}, \ldots w_d^{(i)})\}_{i=1}^m$}
  \STATE{$S'= \emptyset$}
  \NEWLINE
  \FOR{Cada atributo atributo $j=1, \ldots, d$}
    \IF{El atributo $j-$ésimo es numérico}
      \STATE{Calcular la varianza $\sigma_j' = \sqrt{\frac{1}{m}\sum_{i=1}^m \left(w_j^{(i)} - \frac{\sum_{i=1}^m w_j^{(i)}}{m} \right)^2}$}
    \ENDIF
  \ENDFOR
  \NEWLINE
  \STATE{Hacer $M = \left\lceil T/m \right\rceil$}
  \FOR{$t=1, \ldots, M$}
    \FOR{$i=1,\ldots, m$}
      \STATE{Escoger $r \sim N(0,1)$}
      \FOR{$j=1, \ldots, d$}
         \IF{El atributo $j-$ésimo es numérico}
	   \STATE{$w_j = w_j^{(i)} - \frac{\sigma_j'}{\sqrt{m}} \cdot r$}
	 \ELSE
	   \STATE{Escoger $w_j$ de manera uniforme sobre $\{w_j^{(1)}, \ldots w_j^{(m)}\}$}
	 \ENDIF
      \ENDFOR
      \STATE{$S' = S' \cup \{(w_1, \ldots w_d)\}$}
    \ENDFOR
  \ENDFOR
  \NEWLINE
  \STATE{$S'=$ Escoger $T$ instancias aleatorias de entre $S'$}
  \RETURN{$S'$, ejemplos positivos sintéticos}
\end{algorithmic}
\caption{Algoritmo de \textit{oversampling} RWO}
\label{alg:rwo}
\end{algorithm}

Implementado en R, el cuerpo principal del algoritmo, omitiendo otros detalles de la implementación, queda:

\lstinputlisting[language=R, caption = Cuerpo del algoritmo RWO]{codelst/rwo.R}