\label{ch:algs}
\section{Algoritmo MWMOTE}
El algoritmo \ref{alg:smote} presentado anteriormente presenta el problema de que no incluye detección de
las instancias en la frontera entre ambas clases, que son en muchas ocasiones las más complicadas de aprender y las que más
información aportan, debido a la habitual existencia de \textit{overlaping} entre clases en dichas zonas. Tras SMOTE surgió
Borderline-SMOTE que detectaba instancias en la frontera bajo la tesis de que para una instancia minoritaria en la frontera,
al efectuar el cálculo de sus $k$ vecinos más cercanos, más de la mitad serían instancias negativas, dado que la
clase mayoritaria tiene mayor número de instancias.

La última afirmación presenta problemas básicos como que instancias ruidosas podrían ser consideradas como fronterizas, o que
el valor de $k$ no puede ser el mismo para todas las zonas del espacio, puesto que habrá zonas mucho más infrarrepresentadas
que otras. También podemos añadir que si generamos instancias minoritarias entre dos clústers de instancias positivas (esto 
podría pasar para dos clústeres muy infrarrepresentados y un $k$ alto), la instancia que se generaría quedaría fuera del 
espacio de dichas regiones, podiendo ocasionar ruido en regiones de instancias negativas.

El algoritmo MWMOTE (\textit{Majority Weighted Minority Oversampling Technique}) surge como una modificación de SMOTE para poner 
solución a los anteriores problemas. Sea $\spos = \{x_1, \ldots x_m\}$, clase minoritaria y $\sneg = \{y_1, \ldots y_m\}$, 
clase mayoritaria, con $S= \spos \cup \sneg$ el conjunto de entrenamiento. Se usará KNN con distancia euclídea, donde notaremos
$d(x,y)$ a la distancia euclídea entre $x$ e $y$. Notaremos para una instancia $x\in S$, $NN^{k}(x)\subseteq S$ a su $k$-KNN vecindario, 
esto es, el conjunto de las $k$ más cercanas (con distancia euclídea) a $x$. Notaremos $NN_{+}^k(x)$ a su $k$-KNN vecindario 
positivo, esto es las $k$ instancias más cercanas pertenecientes a $\spos$. Análogamente definimos $NN_{-}^k(x) \subseteq \sneg$ 
como el $k$-vecindario negativo de una instancia.

\begin{algorithm}[H]
\begin{algorithmic}[1]
  \REQUIRE $\spos = \{x_1, \ldots x_m\}$, clase minoritaria
  \REQUIRE $\sneg = \{y_1, \ldots y_m\}$, clase mayoritaria
  \REQUIRE $T$, número de instancias sintéticas deseado
  \REQUIRE $k_{1}$, parámetro de KNN para filtrar ruido de $\spos$
  \REQUIRE $k_{2}$, parámetro de KNN para calcular frontera $U \subseteq \sneg$
  \REQUIRE $K_{3}$, parámetro de KNN para calcular frontera $V \subseteq \spos$
  \REQUIRE $\alpha$, umbral de tolerancia en la cercanía a la frontera
  \REQUIRE $C$, ponderación de la cercanía a la frontera
  \REQUIRE $C_{clust}$, parámetro del \textit{clustering} para determinar el número de clústers
  \STATE{Inicializar $S' = \emptyset$, instancias sintéticas}
  \STATE{Para cada $x\in \spos$, calcular su $k_{1}$ KNN vecindario, $NN^{k_1}(x)$}
  \STATE{Hacer $\spos_f = \spos - \{x\in \spos : NN^{k_1}(x) \cap \spos = \emptyset\}$}
  \STATE{Calcular frontera negativa $U = \underset{x\in \spos_f}{\bigcup} NN_{-}^{k_2}(x)$}
  \STATE{Calcular frontera positiva $V = \underset{x\in U}{\bigcup} NN_{+}^{k_3}(x)$}
  \STATE{Para cada $x\in V$, calcular $P(x) = \sum_{y\in U} I_{\alpha,C}(x,y)$}
  \STATE{Normalizar para cada $x\in V$, $P(x) = \frac{P(x)}{\sum_{z\in V} P(z)}$}
  \STATE{Calcular $T_{clust} = C_{clust} \cdot \frac{1}{|\spos_f|} \sum_{x\in \spos_f} \min_{y\in \spos_f, y\neq x} d(x,y)$}
  \STATE{Calcular $L_1, \ldots L_M\subseteq \spos$ clústers para $\spos$, con umbral $T_{clust}$}
  \NEWLINE
  \FOR{$t=1, \ldots, T$}
    \STATE{Escoger $x\in V$ de acuerdo a la probabilidad $P(x)$}
    \STATE{Seleccionar $y\in L_k$ uniformemente donde $L_k \ni x$ ($L_k$ es el clúster de $x$)}
    \STATE{Seleccionar $r$ en $[0,1]$ de manera uniforme}
    \STATE{$S' = S'\cup \{x + r(y-x)\}$}
  \ENDFOR
  \NEWLINE
  \RETURN{$S'$, ejemplos positivos sintéticos}
\end{algorithmic}
\caption{Algoritmo de \textit{oversampling} MWMOTE}
\label{alg:mwmote}
\end{algorithm}


El conjunto $\spos_f \subseteq \spos$ se construye para filtrar instancias ruidosas (aquellas que no tienen ninguna otra 
de la clase positiva alrededor suyo). $U$ representa la frontera de instancias negativas (habrá que tomar preferiblemente
un $k_2$ pequeño, ya que la densidad de representación de la clase negativa es mucho mayor que la de la positiva). La 
construcción de $V$, o frontera de instancias positivas, debe hacerse con un $k_3$ grande por un motivo antagónico al 
ya expresado para $U$.

MWMOTE tiene 3 objetivos fundamentales: dar mayor peso a instancias minoritarias fronterizas, a instancias pertenecientes
a clústeres de baja representación, y a instancias minoritarias muy cerca de zonas con muchas instancias mayoritarias.

Definimos el peso informativo de $x$ respecto a $y$ como $I_{\alpha,C}(x,y) = C_f(x,y) \cdot D_f(x,y)$, donde si 
$x \notin NN_{+}^{k_3}(y)$, entonces $I_{\alpha,C}w(x,y) = 0$. Caso opuesto, se toma:
\[
  f(x) = \left\{\begin{array}{ll} 
                x &, x\le \alpha \\
                \alpha & \textrm{en otro caso}
               \end{array}\right.,\qquad C_f(x,y) = \frac{C}{\alpha} \cdot f\left(\frac{d}{d(x,y)}\right)
\]

$C_f$ es un factor que mide la cercanía a $y$, es decir, mide la cercanía entre dos instancias fronterizas. Se toma 
$D_f(x,y) = \frac{C_f(x,y)}{\sum_{z\in V} C_f(z,y)}$, factor de densidad, de manera que una instancia perteneciente
a un clúster más compacto tendrá mayor $\sum C_f(z,y)$ que una perteneciente a un clúster muy disperso. El último de los
objetivos de detección de instancias de MWMOTE se cumple al ponderar cada instancia $x$ por la suma 
$\sum_{y\in U} I_{\alpha, C}(x,y)$.

Por último queda explicar el algoritmo de \textit{clustering} que se usa: un algoritmo de \textit{clustering} jerárquico
aglomerativo con enlace medio (esto es, se define $dist(L_i, L_j) = \frac{1}{|L_i||L_j|} \sum_{x\in L_i} \sum_{y\in L_j} d(x,y)$.


\begin{algorithm}[H]
\begin{algorithmic}[1]
  \REQUIRE $\spos = \{x_1, \ldots x_m\}$, clase minoritaria
  \STATE{Inicializar $L_i = \{x_i\}, i=1, \ldots, m$}
  \WHILE{$\inf_{i,j, i\neq j} dist(L_i, L_j) \le T_{clust}$, y quede más de un clúster}
    \STATE{Sean $(i,j) = \argmin_{i,j, i\neq j} dist(L_i, L_j)$}
    \STATE{$L_i = L_i \cup L_j$. Elimina $L_j$}
  \ENDWHILE
  \RETURN{Clústeres $L_1, \ldots L_k, k\le m$}
\end{algorithmic}
\caption{Algoritmo de \textit{clustering} jerárquico}
\label{alg:hclust}
\end{algorithm}

Nótese que a mayor valor del parámetro $C_{clust}$ menor número de clústeres tendremos, y mayor número de instancias tendrán
cada uno.
