Introduciremos el concepto de aprendizaje no-uniforme, relajando la definición de agnósticamente PAC cognoscible. También
proporcionaremos una noción de codificación para las clases de hipótesis numerables, y otro paradigma para aprenderlas, que
nos llevará al principio de la \textit{navaja de Occam} en el contexto del aprendizaje automático.

\begin{definition}[Aprendizaje no-uniforme]
Una clase de hipótesis $H$ sobre $Z=\mathcal{X} \times \mathcal{Y}$ es no-uniformemente PAC cognoscible si existe
un algoritmo $A$ y una función $m_{H}^{NU} : ]0,1[^2 \times H \rightarrow \mathbb{N}$ verificando que dados 
$0 < \varepsilon, \delta < 1$ y $h \in H$, entonces para toda distribución $\dist$ sobre $Z$ y
$m\ge m_{H}^{NU} (\varepsilon, \delta, h)$ se verifica:
\[
  \mprob\bigg[L_{\dist}(A(S)) \le L_{\dist}(h) + \varepsilon\bigg] \ge 1-\delta
\]
\end{definition}


\section{Minimización del riesgo estructural}
Hasta ahora hemos traducido el conocimiento previo sobre el problema como una restricción global en la clase de hipótesis
para la minimización del riesgo empírico. Ahora generalizaremos esto aún más y estableceremos la suposición de que 
$H= \cup_{n\in \mathbb{N}} H_n$, donde cada clase $H_n$ tendrá asignado un peso $w(n)$. 

\begin{fact}
Sea $w : \mathbb{N} \rightarrow [0,1]$ verificando $\sum_{n=1}^\infty w(n) \le 1$, $H \subseteq 2^X$
que puede ser escrita como $H= \cup_{n\in \mathbb{N}} H_n$, donde cada $H_n$ es una clase de Glivenko-Cantelli. 
Escogemos:
\[
  \varepsilon_n : \mathbb{N} \times ]0,1[ \rightarrow ]0,1[, \qquad 
  \varepsilon_n (m,\delta) \in \{\varepsilon \in ]0,1[: m_{H_n}^{UC} (\varepsilon, \delta) \le m\}
\]

Entonces para todo $\delta \in ]0,1[$ y para toda distribución $\dist$ se verifica:
\[
  \mprob\bigg[\forall n\in \mathbb{N}, \forall h\in H_n, |L_{\dist}(h) - L_S(h)| \le 
              \varepsilon_n (m, w(n) \delta)\bigg] \ge 1-\delta
\] 

En particular, tomando $n(h)\in \{n: h\in H_n\}$ para cada $h\in H$:
\[
  \mprob \bigg[\forall h\in H, |L_{\dist}(h) - L_S(h)| \le \varepsilon_{n(h)} (m, w(n(h)) \delta)\bigg] \ge 1-\delta
\]
\label{th:non-uniform-complex}
\end{fact}

  \begin{proof}
   Fijado $n$, por ser $H_n$ de Glivenko-Cantelli:
   \[
     \mprob\bigg[\forall h\in H_n, |L_{\dist}(h) - L_S(h)| \le \varepsilon_n (m, w(n)\delta)\bigg] \ge 1- w(n) \delta
   \]

   Por tanto:
   \begin{align*}
          &\mprob \bigg[\forall n\in \mathbb{N}, \forall h\in H_n, |L_{\dist}(h) - L_S(h)| \le \varepsilon_n (m, w(n) \delta)\bigg]\\
   = 1 -  &\mprob\bigg[\exists n\in \mathbb{N}, \exists h\in H_n, |L_{\dist}(h) - L_S(h)| > \varepsilon_n (m, w(n) \delta)\bigg]\\
   \ge 1- &\sum_{n=1}^{\infty} w(n) \delta \ge 1-\delta
   \end{align*}
  \end{proof}

Este teorema motiva las definiciones de riesgo estructural y minimizador de dicho riesgo.

\begin{definition}[Riesgo estructural]
Sea $H = \cup_{n\in \mathbb{N}} H_n$ donde $H_n$ es de Glivenko-Cantelli y tenemos una función de peso
$w : \mathbb{N} \rightarrow [0,1]$, $\sum_n w(n) \le 1$. Dado $h\in H$, escogemos:
\[
  \varepsilon_n (m,\delta) = \min_{\varepsilon \in ]0,1[}\{m_{H_n}^{UC} (\varepsilon, \delta) \le m\}, \quad n(h) = min\{n: h\in H_n\}
\]

Dado $S\in (X\times Y)^m$, se llama riesgo estructural de $h\in H$, respecto a $\varepsilon_n, w$:
\[L_S(h) + \varepsilon_{n(h)} \left(m, w(n(h))\frac{1}{m}\right)\]
\end{definition}

Por sencillez de las demostraciones, supondremos que existe el mínimo de $\{m_{H_n}^{UC}(\epsilon, \delta)\}$ y por tanto
podemos definir $\varepsilon_n$.

\begin{definition}[Minimizador del riesgo estructural, SRM]
Decimos que un algoritmo $A: \underset{m\in \mathbb{N}}{\bigcup} (X\times Y)^m \rightarrow H$ es un $SRM$ 
(\textit{Structural Risk Minimizer}) si busca una hipótesis cuyo error estructural sea mínimo, suponiendo que
dicho mínimo tiene sentido. Dado $S \in (X\times Y)^m$ devolverá:
\[
  A(S) \in \argmin_{h\in H} L_S(h) + \varepsilon_{n(h)} \left(m, w(n(h))\frac{1}{m}\right)
\]
\end{definition}

Es conocido que $\sum_{n=1}^\infty \frac{1}{n^2} = \frac{\pi^2}{6}$. Usando este hecho
y el anterior resultado, podemos dar condiciones suficientes para que un $SRM$ haga a una clase no-uniformemente
cognoscible.

\begin{theorem}
Sea $H$ una clase de hipótesis verificando $H = \cup_{n\in \mathbb{N}} H_n$ donde cada 
$H_n$ es de Glivenko-Cantelli. Sea $w : \mathbb{N} \rightarrow [0,1]$ dada por $w(n) = \frac{6}{(\pi n)^2}$. 
Entonces $H$ es no-uniformemente cognoscible para cualquier algoritmo $A$ que sea un $SRM$. Además se verifica:
\[
  m_{H}^{NU} (\varepsilon, \delta, h) \le \left\{m_{H_{n(h)}}^{CU} \left(\varepsilon/2, \frac{3 \delta}{(\pi n(h))^2} \right), 
  \left\lceil\frac{2}{\delta}\right\rceil \right\}
\]
\label{th:srm-suficientes}
\end{theorem}
  \begin{proof}
   Fijamos $\bar{h}\in H$ y $0 < \delta, \varepsilon < 1$. Sea:
   \[
     m \ge \max\left\{m_{H_{n(\bar{h})}}^{CU} \left(\varepsilon/2, \frac{3 \delta}{(\pi n(\bar{h}))^2} \right), 
         \left\lceil\frac{2}{\delta}\right\rceil \right\}
   \]
   Claramente $\varepsilon_{n(\bar{h})}\left(m, w(n(\bar{h})) \frac{1}{m}\right) \le \varepsilon/2$.

   Usando el hecho de que $\sum_{n\ge 1} w(n) = 1$, por proposición \ref{th:non-uniform-complex} se verifica:    
   \begin{equation}
     \mprob \bigg[\forall h\in H |L_{\dist}(h) - L_S(h)| \le \varepsilon_{n(h)} \left(m, w(n(h)) \frac{1}{m}\right)\bigg] 
     \ge 1-\frac{1}{m} \ge 1 - \frac{\delta}{2}
   \label{ineq:srm-h}\tag{\textrm{$\ast$}}
   \end{equation}
   
   Por ser $A$ un $SRM$, entonces dado $S\in (X\times Y)^m$:
   \begin{align}
     L_{S}(A(S)) + \varepsilon_{n(A(S))} \left(m, w(n(A(S)))\frac{1}{m}\right)
                     &\le L_S(\bar{h}) + \varepsilon_{n(\bar{h})} \left(m, w(n(\bar{h}))\frac{1}{m}\right) \nonumber\\
                     &\le L_S(\bar{h}) + \frac{\varepsilon}{2}
     \label{ineq:srm}\tag{\textrm{$\ast$$\ast$}}
   \end{align}
   
   A partir de \eqref{ineq:srm-h} y \eqref{ineq:srm}, deducimos:
   \begin{align}
                     &\mprob \bigg[L_{\dist}(A(S)) \le L_{S}(\bar{h}) + \varepsilon/2\bigg] \ge 1-\frac{\delta}{2} \nonumber\\
     \Leftrightarrow &\mprob \bigg[L_{\dist}(A(S)) > L_{S}(\bar{h}) + \varepsilon/2\bigg] \le \frac{\delta}{2}
     \label{ineq:LDbound}\tag{\textrm{$\dagger$}}
   \end{align}
   
   Además, usando convergencia uniforme, con probabilidad menor o igual que $\delta/2$:
   \begin{equation}
     L_S(\bar{h}) > L_{\dist}(\bar{h}) + \varepsilon/2
     \label{ineq:Lhbound}\tag{\textrm{$\dagger$$\dagger$}}
   \end{equation}

   Uniendo \eqref{ineq:LDbound} y \eqref{ineq:Lhbound} deducimos, por subaditividad:
   \[
     \mprob \bigg[L_{\dist}(A(S)) > L_{S}(\bar{h}) + \varepsilon/2 \bigvee L_S(\bar{h}) > L_{\dist}(\bar{h}) + \varepsilon/2\bigg] \le \delta
   \]
   
   Es decir:
   \[
     \mprob \bigg[L_{\dist}(A(S)) \le L_{\dist}(\bar{h}) + \varepsilon \bigg] \ge 1 - \delta
   \]
  \end{proof}

\begin{theorem}[Caracterización de aprendizaje no-uniforme]
Una clase de hipótesis $H \subseteq 2^X$ es no-uniformemente cognoscible sii $H = \bigcup_{n\in\mathbb{N}} H_n$ donde
cada $H_n$ es APAC cognoscible.
\end{theorem}
  \begin{proof}
   Supongamos $H = \cup_{n\in \mathbb{N}} H_n$ donde cada $H_n$ es agnósticamente PAC 
   cognoscible. Por teorema fundamental de aprendizaje PAC, \ref{th:fundamental}, cada clase $H_n$ tiene la propiedad de 
   convergencia uniforme, y podemos aplicar el teorema anterior.

   En el sentido opuesto, sea $H$ no-uniformemente cognoscible usando un cierto algoritmo $A$. Fijamos $\delta < \frac{1}{7}$ y
   definimos:
   \[
     H_n = \{h \in H : m_{H}^{NU}(1/8, \delta, h) \le n\}
   \]
   para todo $n$ natural. Claramente $H = \cup_{n\in \mathbb{N}} H_n$. 
   
   Supongamos $VC(H_n) = \infty$ para algún $n\in \mathbb{N}$. Entonces por teorema de No Free Lunch, \ref{th:nfl2}, 
   tendríamos que existe $\dist$ una distribución sobre $X$ verificando:
   \[
     \exists \bar{h}\in H_n: L_{\dist}(\bar{h}) = 0 \qquad\textrm{y}\quad \mprob \left[ L_{\dist}(A(S)) > \frac{1}{8} \right] \ge \frac{1}{7}
   \]
   lo cual para $m > m_H^{NU}(\frac{1}{8}, \delta, \bar{h})$ constituiría una contradicción.
  \end{proof}
  
\begin{corollary}
 La noción de cognoscibilidad no-uniforme es más fuerte que la APAC cognoscibilidad.
\end{corollary}
\begin{proof}
 Dada $H$ APAC cognoscible, por el teorema anterior es no-uniformemente cognoscible.
 
 Por otro lado $H = \{h:\mathbb{R} \rightarrow \{-1,1\} : h^{-1}(1) \textrm{ es finito}\}$ es no-uniformemente cognoscible,
 pero no es APAC cognoscible. Veámoslo.
 
 Escribimos $H_n = \{h\in H: |h^{-1}(1)| \le n\}$. Es claro que $H=\bigcup_{n\in \mathbb{N}} H_n$. Además, $VC(H_n) = n$,
 porque no podríamos fragmentar conjuntos de tamaño mayor que $n$, al ser $h^{-1}(1)$ de cardinal $n$ como mucho. Es trivial
 ver que sí podemos fragmentar conjuntos de tamaño $n$. Luego por teorema fundamental del PAC, \ref{th:fundamental}, 
 cada $H_n$ sería APAC cognoscible, y el teorema anterior nos dice que $H$ es no-uniformemente cognoscible.
 
 Pero $VC(H) = \infty$, porque $VC(H) \ge VC(H_n) = n$ para $n\in \mathbb{N}$ arbitrario, luego $H$ no es APAC cognoscible.
\end{proof}
  
\section{Minimización de longitud descriptiva}
\begin{definition}[Lenguaje de descripción de hipótesis]
 Sea $H$ una clase de hipótesis y $\Gamma$ un conjunto finito de símbolos, que llamaremos alfabeto. Notaremos 
 al lenguaje generado por $\Gamma$:
 \[
   \Gamma^{\ast} = \{(\gamma_1, \ldots, \gamma_m): \gamma_i \in \Gamma, m\in \mathbb{N}\}
 \]
 donde la longitud de una palabra $\gamma = (\gamma_1, \ldots, \gamma_m) \in \Gamma$ será $m$ y lo notaremos $|\gamma| = m$.
 
 Un lenguaje de descripción para $H$ será una asignación $d: H \rightarrow \Gamma^{\ast}$, libre de prefijos,
 esto es, dados $h,h'\in H$ distintos, con $|d(h)| \le |d(h')|$, los $|d(h)|$ primeros símbolos de $d(h')$
 no podrán ser $d(h)$ (al menos no en ese orden).
 
 Dado un lenguaje de descripción de $H$, a saber, $d$, notaremos $|h| = |d(h)|$.
\end{definition}

Por simplicidad, consideraremos $\Gamma = \{a,b\}$ en lo que sigue.

\begin{lemma}[Desigualdad de Kraft]
 Si $\mathcal{L}\subseteq \Gamma^{\ast}$ es un conjunto de palabras provenientes de un lenguaje libre de prefijos, entonces:
 \[
   \sum_{\alpha \in \mathcal{L}} \frac{1}{2^{|\alpha|}} \le 1
 \]
 \label{lemma:kraft}
\end{lemma}
  \begin{proof}
   Como las palabras del lenguaje son libres de prefijos, podríamos representar $\mathcal{L}$ con un árbol binario, donde
   cada palabra sería el camino más corto desde una hoja a la raíz del árbol ($\emptyset$). Recíprocamente, cualquier árbol binario estaría
   representando un lenguaje libre de prefijos, si consideramos la misma regla para formar palabras.
   
   Hacemos la demostración sobre la altura del árbol, identificando cada hoja de un árbol, $(T,E)$, con una palabra.
   
   \begin{subenv}
    Para un árbol de altura $0$, tendríamos la palabra vacía, que tendría longitud $0$, y por tanto $\frac{1}{2^0} = 1$.
    
    Supuesto que se cumple para árboles de hasta altura $n$, y sea un árbol de altura $n+1$, a saber $(T,E)$. Sean $J$ sus hojas.
    Si podásemos las ramas de longitud $n+1$, obtendríamos un árbol binario de altura $n$, con hojas $J'$ por lo que cumpliría:
    \[
      \sum_{l'\in J'} \frac{1}{2^{|l'|}} \le 1
    \]
    Pero la rama correspondiente a una hoja $l'$ podría dividirse como mucho en dos ramas para volver a tener la rama original $l$. Si 
    se dividiese en una, tendríamos $|l| > |l'|$ y por tanto $\frac{1}{2^{|l|}} \le \frac{1}{2^{|l'|}}$. Si se dividiese en dos, tendríamos
    dos factores que valdrían la mitad, y al sumarse, $\frac{1}{2^{|l'|}}$. Si no se dividiese en ninguna, no se alteraría
    el factor correspondiente.
   \end{subenv}
  \end{proof}

\begin{fact}
 Sea $H \subseteq 2^X$ una clase de hipótesis numerable, un lenguaje de descripción $d:H \rightarrow \{a,b\}^{\ast}$. Entonces
 para cualquier $m\in \mathbb{N}$, cualesquiera $0 < \delta, \epsilon < 1$ y una distribución $\dist$ sobre $X$ arbitraria, 
 se cumpliría:
 \[
   \mprob\bigg[\forall h\in H, L_{\dist}(h) \le L_S(h) + \sqrt{\frac{|h| + \log(2/\delta)}{2m}} \bigg] \ge 1 - \delta
 \] 
 \label{fact:occam}
\end{fact}
\begin{proof}
 Sea $H = \bigcup_{n\in \mathbb{N}} \{h_n\}$. 
 
 Por proposición \ref{fact:finitas-gc} tendríamos que $m^{CU}_{\{h_n\}} \le \left\lceil \frac{\log(2/\delta)}{2\varepsilon^2} \right\rceil$
 para cada $n\in \mathbb{N}$ arbitrario, respecto de la función de pérdida $0-1$.
 
 Luego $\varepsilon_n(m,\delta) = \sqrt{\frac{\log(2/\delta)}{2m}}$ y el $SRM$ estaría bien definido. Tomando $w(n) = \frac{1}{2^{|h_n|}}$, donde
 $|h_n|$ es la longitud dada por el lenguaje de descripción $d$, el lema \ref{lemma:kraft} de Kraft nos diría 
 $\sum_{n \ge 1} w(n) \le 1$.
 
 Aplicando la proposición \ref{th:non-uniform-complex} tendríamos que:
 \[
   \mprob\bigg[\forall h\in H, L_{\dist}(h) \le L_S(h) + \sqrt{\frac{-\log(1/2^{|h|}) + \log(2/\delta)}{2m}} \bigg] \ge 1 - \delta
 \]
 
 Pero como $-\log(1/2^{|h|}) = |h|\log(2) \le h$, y $\sqrt{(\cdot)}$ es creciente, por subaditividad:
 \[
   \mprob\bigg[\forall h\in H, L_{\dist}(h) \le L_S(h) + \sqrt{\frac{|h| + \log(2/\delta)}{2m}} \bigg] \ge 1 - \delta
 \]
\end{proof}

Este teorema, adaptación de \ref{th:non-uniform-complex} al caso de $H$ numerable y existencia de un lenguaje de descripción,
motiva que adaptemos también el $SRM$ a dicho paradigma.

\begin{definition}[Minimizador de la longitud descriptiva]
Decimos que un algoritmo $A: \underset{m\in \mathbb{N}}{\bigcup} (X\times Y)^m \rightarrow H$ es un $DLM$ 
(\textit{Description Length Minimizer}) si dado $S \in (X\times Y)^m$ entonces:
\[
  A(S) \in \argmin_{h\in H} L_S(h) + \sqrt{\frac{|h| + \log(2m)}{2m}} 
\]
\end{definition}

El teorema \ref{th:srm-suficientes} nos dice que para una clase $H$ numerable y $d$ su lenguaje de descripción, un $DLM$
hace a $H$ no-uniformemente cognoscible.

\subsection{Navaja de Occam}
De la proposición \ref{fact:occam} se desprende que tomando dos $h, h' \in H$ con igual $L_S(h)$, sería preferible tomar siempre
la hipótesis de mínima longitud de descripción $|h|$. Este principio es conocido como \textit{navaja de Occam}, y establece
que una hipótesis sencilla (de descripción corta) tiende a ser más válida que una complicada (de descripción más larga).

\section{Aprendizaje PAC vs aprendizaje no-uniforme}
En términos de error, ambos paradigmas proporcionan una cota del error cometido, que se cumple con cierta probabilidad. Ahora
bien, en paradigma no-uniforme no podemos conocer a priori cuántos ejemplos son necesarios para aprender la mejor hipótesis,
mientras que en aprendizaje PAC sí le pedimos esta condición al algoritmo y la función de complejidad muestral.

Puesto que ambos paradigmas proporcionan cotas para $err_{est}$, sabemos qué parte del error corresponde a la aproximación
y cuál a la estimación. Si el error de aproximación es muy alto, siempre podemos intentar aprender con otra clase $H$ en PAC
o modificar el esquema de asignación de pesos $w(n)$ en aprendizaje no-uniforme.

La pregunta ¿cómo aprender? es altamente dependiente del problema. Ambos paradigmas nos permiten usar conocimiento
previo para que un problema sea resoluble: PAC restringiendo $H$ y aprendizaje no-uniforme escogiendo $w(n)$. El aprendizaje
no-uniforme es más flexible que el aprendizaje PAC, pero proporciona menos garantías en base al número de ejemplos que necesitamos
para aprender.