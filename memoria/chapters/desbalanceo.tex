\section{Problema de desbalanceo}
Se considerará en toda la sección de informática clasificación binaria, donde tenemos un dominio $X$, y un conjunto de etiquetas 
$Y=\{-1,1\}$. Llamamos $Z = X\times Y$. También consideraremos que los datos siguen una distribución dada por una función de 
probabilidad prefijada pero no conocida por el algoritmo, a saber, $\dist=(\Sigma, P)$. 

Por simplicidad, podemos considerar siempre $\Sigma = \mathcal{P}(X)$, por lo que identificaremos en lo que sigue $\mathcal{D} \equiv P$.
El problema de clasificación binaria consiste en buscar, dado $S= \left\{(x_1, y_1), \ldots (x_m, y_m)\right\}$ un 
conjunto de entrenamiento, un clasificador de alguna clase $H$, $h$, verificando que $L_{P}(h)$ sea lo más pequeño posible.
Nótese que estamos relajando las hipótesis de \ref{sec:clas-binaria} al considerar la distribución de probabilidad como prefijada.

\begin{definition} \textbf{Instancias positivas, negativas}

Llamamos instancias positivas a $Z^{+} = \{(x,y)\in Z: y=1\}$.

LLamamos instancias negativas a $Z^{-} = \{(x,y)\in Z: y=-1\}$.

Una notación equivalente se aplica a $S$. Además llamamos $\ppos = P(Z_{+})$ y $\pneg = P(Z_{-})$.
\end{definition}


\begin{definition} \textbf{Desbalanceo entre clases}

 Decimos que un problema de clasificación binaria es desbalanceado (entre clases) si se verifica que tomando 
 $\nneg = |S^{-}|$ y $\npos = |S^{+}|$, entonces $\npos < \nneg$.
 
 Llamamos ratio de desbalanceo a $r(S)=\frac{\nneg}{\npos}$.
\end{definition}

Por convención se consideran problemas de desbalanceo donde $r(S) \gg 1$, tal y como se afirma en \cite{he2009}. Asimismo,
en estos problemas nos interesa especialmente mejorar el error de los clasificadores sobre la clase minoritaria o positiva.

En muchas ocasiones, al realizar análisis de datos no se tiene en cuenta el desbalanceo existente entre clases, lo
cual hace que los algoritmos no reflejen adecuadamente la distribución de las clases. Para ciertos conjuntos de datos,
donde el número de instancias positivas con respecto a las negativas es muy pequeño, un clasificador podría simplemente
etiquetar todas las instancias como negativas, y obtener un error global pequeño, pero no acertaría en ninguna instancia
positiva. Si los datasets empleados representaran enfermedades, por ejemplo, como es el caso de muchos de los
disponibles en el repositorio \citepalias{keel}, algunos de los cuales se han integrado dentro del paquete, 
nos interesaría encontrar mecanismos para disminuir el error sobre la clase positiva (la de las instancias correspondientes
a la enfermedad).

El desbalanceo puede ser:

\begin{enumerate}[i]
  \item \textbf{Intrínseco}, cuando $\ppos < \pneg$, y por tanto la cantidad de datos de cada clase que extraemos viene 
  verdaderamente determinada por la distribución real de los datos.

  \item \textbf{Extrínseco}, si $\ppos \ge \pneg$ pero sin embargo para nuestro conjunto de entrenamiento $S$, 
  tenemos que $\npos < \nneg$. Esto quiere decir que el desbalanceo no viene determinado por la verdadera distribución de
  los datos, sino por factores como por ejemplo la complejidad para extraer la misma cantidad de datos de la misma clase.
\end{enumerate}

También podemos decir existe desbalanceo \textbf{intra clases} cuando además la clase minoritaria está repartida en varias 
regiones no conexas, a saber $\sneg \supset A_1, \ldots A_m$ donde $|A_1| \le \ldots \le |A_m|$, con algún menor o igual estricto.
Esto dificulta aún más la correcta clasificación de dichas instancias, porque aparecen zonas con instancias positivas 
pobremente representadas. A estas zonas de baja representación las llamamos instancias raras. Nótese que hay que distinguir
el concepto de instancia rara de otro concepto que surge en ciencia de datos: el concepto de ruido, instancias aisladas completamente.
   
La aparición de instancias raras no sólo se debe al desbalanceo de clases, sino que se puede deber a un amplio rango de factores 
que afectan a la complejidad de los datos: \textit{overlapping} ($\sneg_x \cap \spos_x \neq \emptyset$), falta de 
representatividad en los datos, \textit{small disjuncts}, etc. En general, los clasificadores intentan aprender a partir de una clase 
creando reglas (conceptos) disjuntas que agrupen a instancias. Como consecuencia de la infrarrepresentación de instancias, podemos 
tener reglas que cubran una pequeña porción de las instancias de la clase positiva. Estas instancias conforman lo que llamamos 
un \textit{small-disjunct}. Los \textit{small-disjuncts} no sólo afectan a la clase minoritaria, sino que pueden darse también 
dentro de la mayoritaria, aunque la mayor densidad de datos de esta clase hace que el efecto no sea tan agravado o 
sea una situación menos frecuente. De hecho hay estudios como \cite{jo2004} que sostienen que no sólo el desbalanceo de clases
es el responsable de la pérdida de acierto de los clasificadores, sino que los \textit{small disjuncts} son un factor muy a tener en
cuenta en la degradación del rendimiento de los mismos.

\imgcaption{./imgs/desbalanceo.png}{Ejemplo de dataset con desbalanceo}{0.85}

\section{Técnicas para aprendizaje con desbalanceo}
\subsection{Oversampling/Undersampling}
El \textit{oversampling} consiste en generar instancias sintéticas etiquetadas como positivas, $E$, normalmente 
respetando la distribución de $\spos$, intentando generar conexos a partir de instancias raras, etc. Se lleva a cabo la 
tarea de clasificación usando el conjunto de entrenamiento $S' = S\cup E$. 

El \textit{undersampling} consiste en escoger un subconjuto $E \subseteq \sneg$, de manera que $S' = S\setminus E$ 
y tenemos un problema de clasificación con el conjunto de entrenamiento $S'$. Normalmente $E$ se genera de manera que se
intentan eliminar instancias que aportan información redundante dentro del conjunto de instancias negativas.

Suponemos también que tanto el \textit{oversampling} como el \textit{undersampling} se realiza de manera que $r(S') < r(S)$.

La gran ventaja de los métodos de \textit{sampling} respecto a otros como \textit{cost-sensitive} es su flexibilidad, puesto
que podemos aplicárselos a cualquier clasificador.

\subsubsection{Limpieza de instancias en \textit{oversampling}}

Estas técnicas se emplean para limpiar los bordes de las zonas que conectan a ambas clases. El método más reseñable son los
\textit{enlaces Tomek}. Se dice que dos instancias $x_i, x_j$ forman un \textit{enlace Tomek} si 
$d(x_i, x_j) = \argmin_{x_p, x_q} d(x_p, x_q)$. Por tanto si dos instancias corresponden a un \textit{enlace Tomek} significará,
a no ser que correspondan a ruido, que están en la frontera entre las dos clases. Por tanto dichos enlaces pueden usarse para
una vez se ha hecho \textit{oversampling} de la clase minoritaria, ir eliminando iterativamente todos los \textit{enlaces Tomek}
cuyas dos instancias pertenezcan a clases distintas, hasta que los únicos enlaces que resten pertenezcan a la misma clase.

También destaca el método ENN (\textit{Edited Nearest Neighbour}), que consiste en eliminar todas aquellas instancias tales
que al calcular sus $k$ vecinos más cercanos, haya una mayoría pertenecientes a la otra clase de entre los mismos.

\subsection{Aprendizaje \textit{cost sensitive}}

El \textit{framework} de aprendizaje \textit{cost sensitive} implica que no a todas las instancias le asignamos el mismo error
al aprender (como podría ser una función de pérdida $0-1$ como las descritas en \ref{def:zero-one-loss}). Una técnica para usar
funciones de pérdida aplicadas al problema de desbalanceo consiste en usar una matriz $\{C_{i,j}\}_{i,j\in \{-1,1\}}$, 
donde $C_{i,j}$ reprensenta el coste de asignar a la una instancia de clase $i$, la clase $j$, siendo $C_{i,i} = 0$; es habitual
tomar $C_{1,-1} > C_{-1,1}$ para dar más importancia a la clase positiva.

Dentro de este paradigma destacan métodos de \textit{boosting} y de \textit{ensamble} de clasificadores. También es habitual
encontrarnos algoritmos clásicos modificados para implementar \textit{cost sensitive} hacia la clase minoritaria.

\section{Medidas de la bondad del aprendizaje}

\begin{table}[H]
  \centering
  \begin{tabular}{C{3cm}|C{3cm}}
  $VP$ & $FP$\\
  \hline
  $FN$ & $VN$\\
  \end{tabular}
  \caption{Matriz de confusión}
\end{table}

Para definir medidas de bondad necesitaremos tener en cuenta la matriz de confusión, donde $VP$ (verdaderos positivos) 
representará el número de instancias positivas clasificadas como positivas; $FP$ (falsos positivos) representará las instancias
verdaderamente negativas pero que han sido clasificadas como positivas; $FN$ (falsos negativos) reprensetará instancias 
positivas que han sido clasificadas como negativas; y $VN$ (verdaderos negativos) reprensentará instancias negativas
correctamente clasificadas como negativas. Obsérvese además $\npos = VP + FN$ y $\nneg = FP + VN$, con $N = \npos + \nneg$.

La medida de bondad habitual en clasificación es el acierto o exactitud: $e = \frac{VP + VN}{N}$. Esta medida presenta el
problema en clasificación desbalanceada de que podríamos tener una exactitud muy alta, pero un acierto nulo sobre la clase 
minoritaria. Necesitamos medidas que prioricen el acierto o al menos lo ponderen más en clasificación desbalanceada.

A tal efecto, podemos definir la precisión como $p = \frac{VP}{VP + FP}$, esto es, el nivel de acierto sobre las instancias 
clasificadas como positivas. También tenemos en cuenta la sensibilidad, $s = \frac{VP}{\npos}$ que cuantifica las instancias
positivas verdaderamente etiquetadas como tales, es decir, lo completa que ha sido la clasificación en términos de la clase
minoritaria, reteniendo la información necesaria para clasificar las instancias positivas como tales.

También se define el $F$-score como $F_{\beta} = \frac{1 + \beta^2}{\frac{\beta^2}{s} + \frac{1}{p}}$ como una media 
ponderada de la sensibilidad y la precisión, donde cuantificamos la precisión con un factor $\beta$ de importancia sobre 
la sensibilidad. De hecho en el caso $\beta = 1$ tenemos una media armónica de precisión y sensibilidad. También puede 
expresarse como $F_{\beta} = \frac{(1+\beta^2) VP}{(1+\beta^2) VP + \beta^2 FN + FP}$.

Análogamente al $F$-score podemos definir la media geométrica de la precisión y la sensibilidad $G = \sqrt{s\cdot p}$, que
cuantificaría ambas medidas por igual, siendo menos sensible a que una de ellas sea exageradamente buena y la otra no.

Como última medida que vamos a considerar, tenemos el $AUC$ (\textit{area under the curve}). Para entender lo que es el $AUC$
necesitamos antes definir la curva $ROC$ (\textit{receiver operating characteristic curve}), que se obtiene al representar en el eje
de abcisas la tasa de falsos positivos o (1-especificidad) $\frac{VP}{\npos}$ , y en el eje de ordenadas la tasa de verdaderos positivos
o sensibilidad del clasificador, de manera que nos queda un punto en el espacio $[0,1]\times [0,1]$. La curva $ROC$ se obtiene al unir 
todos los puntos de los que dispongamos para un clasificador (por ejemplo si hacemos una validación cruzada, o si consideramos distribuciones
acumuladas sobre los datos). Intuitivamente, cuanto más cerca del punto $(0,1)$ (acertamos todos los ejemplos verdaderamente positivos,
sin ningún falso positivo), mucho mejor $AUC$ obtendremos. La bisectriz $y=x$ marca el clasificador aleatorio (equivalente a lanzar 
una moneda al aire) que asigna la mitad de instancias a cada clase.

\imgcaption{./imgs/ROC.png}{Curva ROC}{0.85}

\section{Estado del arte del \textit{oversampling}}
A día de hoy, el algoritmo SMOTE, propuesto por \citealt{chawla02} sigue siendo una de las referencias en el campo del 
\textit{oversampling} para clasificación desbalanceada, dada su idea simple pero eficaz de rellenar zonas visuales no 
cubiertas con instancias a partir de instancias generadas en la recta que une otras dos. Damos una breve descripción
del algoritmo.

\begin{algorithm}[H]
\begin{algorithmic}[1]
  \REQUIRE $\spos = \{x_1, \ldots, x_m\}$, instancias positivas
  \REQUIRE $T$, número de instancias sintéticas deseado
  \REQUIRE $k$, parámetro de KNN para crear instancias sintéticas
  \STATE{Inicializar $S' = \emptyset$, instancias sintéticas}
  \STATE{Calcular los $k$ vecinos más cercanos en $\spos$ para cada de $x\in \spos$} 
  \STATE{Toma $M = \left\lceil T/|\spos| \right\rceil$}
  \NEWLINE
  \FOR{$x\in \spos$}
    \FOR{$m=1, \ldots, M$}
      \STATE{Escoger un vecino más cercano para $x$, a saber, $y$}
      \STATE{Seleccionar $r$ en $[0,1]$ de manera uniforme}
      \STATE{$S' = S'\cup \{x + r(y-x)\}$}
    \ENDFOR
  \ENDFOR
  \STATE{Hacer $S' = T$ instancias escogidas aleatoriamente desde $S'$} 
  \NEWLINE
  \RETURN{$S'$, ejemplos positivos sintéticos}
\end{algorithmic}
\caption{Algoritmo de \textit{oversampling} SMOTE}
\label{alg:smote}
\end{algorithm}
