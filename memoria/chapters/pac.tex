\section{PAC cognoscibilidad}
Proporcionamos a continuación una definición que materializa el concepto de que un algoritmo pueda aprender de los datos. Las
siglas PAC derivan de Probablemente Aproximadamente Correcto.

\begin{definition*} \textbf{PAC cognoscible}

Una clase de funciones $H \subseteq 2^X$ es PAC cognoscible sii existen una función 
$m_{H} : ]0,1[^2\rightarrow \mathbb{N}$, llamada complejidad muestral, y un algoritmo 
$A: \underset{m\in \mathbb{N}}{\bigcup} (X\times Y)^m \rightarrow 2^X$ verificando que para todo
$0 < \epsilon, \delta < 1$, para toda distribución $\mathcal{D}$ sobre $X$ y para toda función de 
verdadero etiquetado $f\in H$, dados $m \ge m_H(\epsilon, \delta)$ y $S\sim \mathcal{D}^m$:

\[\mprob \bigg[L_{\mathcal{D},f}(A(S)) \le \epsilon \bigg] \ge 1-\delta\]
\end{definition*}

Llamamos a $(1-\delta)$ \textbf{confianza} de la predicción y a $(1-\epsilon)$ \textbf{exactitud}. Estos dos parámetros 
explican el nombre aproximadamente ($\equiv$ confianza) correcto ($\equiv$ exactitud).

La definición de PAC cognoscibilidad nos exige que exista un algoritmo y una función a la que proporcionándole unos requisitos
de confianza y exactitud, nos indique un tamaño del conjunto de entrenamiento tal que el algoritmo pueda aproximar cualquier
distribución y etiquetado, siempre que le ofrezcamos como entrada un conjunto de entrenamiento que satisfaga los 
requisitos de tamaño.

Consideraremos $m_{H}$ única en el sentido de que para cada $(\delta, \epsilon)$ nos devuelva el menor natural
verificando las hipótesis del enunciado.

\begin{theorem*} \textbf{Las clases finitas son PAC cognoscibles}

Sea $H \subseteq 2^{X}$ finito. Entonces $H$, bajo hipótesis de factibilidad, es PAC cognoscible con:

\[m_H(\epsilon, \delta) \le \left\lceil \frac{1}{\epsilon}log \left(\frac{|H|}{\delta} \right) \right\rceil\]
\end{theorem*}

  \begin{proof}
  Fijamos $\epsilon \in ]0,1[$. Sea una distribución $\mathcal{D}$ sobre $X$, $m\in \mathbb{N}$ y una función de verdadero 
  etiquetado $f\in H$, tomamos la clase de hipótesis ``malas'':

  \[H_B = \{h\in H: L_{\mathcal{D},f}(h) > \epsilon\}\]

  Sea $A$ un $ERM_{\mathcal{H}}$, entonces:

  \[\mprob [L_{\mathcal{D},f}(A(S)) > \epsilon] \le \mprob 
  [\exists h\in H_B : L_S(h) = 0] \le \sum_{h\in H_B} P [L_S(h) = 0] \]

  La primera desigualdad viene dada por la proposición \ref{fact:ermh}. La segunda, por subaditividad, puesto que:
  
  \[\mprob [\exists h\in H_B : L_S(h) = 0] = \mprob \left(\underset{h\in H_B}{\bigcup} [L_S(h) = 0]\right)\]

  Además, fijada $h\in H_B$, como $L_{\mathcal{D},f}(h) > \epsilon$:

  \begin{align*}
  \mprob[L_S(h) = 0] = \mprob[h(x_i) = f(x_i), i =1,\ldots m\}] =\\
  = \prod_{i=1}^m \prob[h = f] = \prod_{i=1}^m (1 - L_{\mathcal{D},f}(h)) \le (1-\epsilon)^m \le e^{-\epsilon m}
  \end{align*}


  Las dos desigualdades probadas, junto a la hipótesis del enunciado, y usando $H_B \subseteq H$ dan lugar a:

  \[\mprob[L_{\mathcal{D},f}(h_S) > \epsilon] \le |H|e^{-\epsilon m}\]
  
  Y basta acabar haciendo encontrando $m\in \mathbb{N}$ tal que $|H|e^{-\epsilon m} \le \delta$.
  \end{proof}

¿Hay ejemplos de clases infinitas PAC cognoscibles? La respuesta es afirmativa. Veamos un ejemplo.

\begin{example}
  \begin{definition*} \textbf{Clasificadores de rectángulo}

  La clase de clasificadores de rectángulo en el plano se define por:

  \[H^2_{rec} = \{ h_{a,b,c,d}: a\le b, c\le d\}\]

  donde $h_{a,b,c,d} = \mathds{1}_{[a,b]\times [c,d]}$
  \end{definition*}


  \begin{fact} \textbf{$H_{rec}^2$ es PAC cognoscible}

  Bajo hipótesis de factibilidad, los rectángulos son PAC cognoscibles
  \end{fact}

    \begin{proof}
    Haremos la prueba tratando los clasificadores como conjuntos, en virtud de la biyección canónica \eqref{sec:hip-con}
    
    Sea $A$ el algoritmo que devuelve el rectángulo más pequeño que engloba a todos los ejemplos positivos del conjunto 
    de entrenamiento $S$.

    Por hipótesis de factibilidad, $\exists R^\ast = [a,b]\times [c,d]$ tal que $L_{\mathcal{D},f}(R^\ast) = 0$, y por 
    proposición \ref{fact:ermh} se tiene $\mprob \bigg[L_S(R^\ast) = 0 \bigg] = 1$. 
    Dado por tanto $S \in (X\times Y)^m$ tal que $L_S(R^\ast) = 0$, claramente $x \in A(S)$ para 
    todo $(x,1)\in S$. Por tanto $A(S) \subseteq R^\ast$, y se deduce $x \notin A(S)$ para 
    cualquier $(x,0)\in S$. Por tanto, $A$ es un ERM.

    Fijamos $1 > \epsilon, \delta > 0$.

    Tomamos los rectángulos:
    
    \begin{align*} 
    R_1 = [a,b^{\ast}] \times [c,d], \qquad R_2= [a^{\ast},b] \times [c,d] \\ 
    R_3=[a,b] \times [c,d^{\ast}],   \qquad R_4=[a,b] \times [c^{\ast},d]     
    \end{align*}
 
    verificándose $R_i \subseteq R^\ast$ y $L_{\mathcal{D},f}(R_i) \le \frac{\epsilon}{4}$. Llamamos 
    $\bar{R}_i = R\setminus R_i$
    
    Supongamos s.p.g que $L_{\mathcal{D},f}(R_i) > 0$ para todo $i\in\{1,\ldots 4\}$ (caso opuesto 
    bastaría hacer la demostración para los rectángulos para los que no fuese nulo). Llamamos 
    $\bar{\epsilon} = \max_i L_{\mathcal{D},f}(R_i)$ y $\widetilde{\epsilon} = \min_i L_{\mathcal{D},f}(R_i)$.

    Fijado $S\in (X\times Y)^m$, $R=A(S)$, supongamos $\forall i : R \cap \bar{R_i} \neq \emptyset$. Entonces:

    \[L_{\mathcal{D},f}(R) = \prob [h_R \neq f] \le \prob 
    \left(\underset{i}{\cup} [h_R \neq f] \cap \bar{R}_i\right) \le \prob \left(\underset{i}{\cup} 
    \bar{R}_i\right) \le 4 \bar{\epsilon}\]

    La demostración acaba probando que:

    \[\mprob [\exists i : A(S)\cap \bar{R}_i = \emptyset] \le \sum_{i=1}^4 
    \mprob[A(S)\cap \bar{R}_i = \emptyset] \le 4 \left(1-\widetilde{\epsilon} \right)^m \le 4e^{-\widetilde{\epsilon}m/4}\]

    y tomando $m > \frac{4}{\widetilde{\epsilon}} log \left( \frac{4}{\delta} \right)$ llegaríamos al resultado buscado.
    \end{proof}
\end{example}


Nótese que las condiciones exigidas, cumplir la hipótesis de factibilidad y que la hipótesis devuelta deba estar en $H$, 
son muy fuertes. En contraposición, esta definición tiene la virtud de que el tamaño muestral es independiente de la 
distribución. Relajaremos esta definición con el concepto de PAC agnóstico (APAC)


\section{APAC cognoscibilidad}

Vamos a relajar tanto la definición de cognoscibilidad como las condiciones sobre las que aprender 
(factibilidad por ejemplo, pérdidas arbitrarias, etc).

\begin{definition} \textbf{Función de pérdida}
Dados una clase $\mathcal{H}$, se denomina función de pérdida de $\mathcal{H}$ sobre $Z$ 
a cualquier función de la forma:

\[l : \mathcal{H} \times Z \rightarrow \mathbb{R}^{+}\]

que verifique que fijada $h\in \mathcal{H}$ arbitrario la función $l(h, \cdot)$ sea medible.
\end{definition}

Modificamos los siguientes puntos de las definiciones básicas en \ref{sec:defs}:

\begin{itemize}
  \item \textbf{Conjunto de etiquetas}: $Y$ arbitrario.

  \item \textbf{Verdadero etiquetado}: No asumimos que exista.

  \item \textbf{Generación de instancias}: Tenemos una distribución $\mathcal{D}$ sobre $Z = X\times Y$. De esta
  manera no asumimos determinismo, y podríamos tener una instancias con varias etiquetas diferentes. Notaremos
  $z = (x,y) \in Z, x\in X, y\in Y$.

  \item \textbf{Error del clasificador}: Lo redefinimos como:
  \[L_{\mathcal{D}}(h) :=  \mathbb{E}_{z\sim \mathcal{D}}[l(h,z)]\]
  
  \item \textbf{Error empírico}: Lo redefinimos como:
  \[L_{S} (h) := \frac{1}{m} \sum_{i=1}^m l(h,z_i)\]
\end{itemize}

Con estos conceptos revisitados, podríamos asegurar que la hipótesis que menor error comete para 
$\mathcal{Y} = \{0,1\}$ y la función de pérdidad $0-1$ es el llamado \textbf{clasificador de Bayes}:

\[f_{\mathcal{D}}(x) = \left\{\begin{array}{ll}
1 & P [y = 1 |x] >= 0.5\\
0 & \quad si \quad no
\end{array}\right.\]

Pero evidentemente asumimos que el algoritmo no tiene acceso a la distribución, sino solo a los datos 
preetiquetados de $S$.


\begin{definition*} \textbf{APAC cognoscible}

Una clase de funciones $\mathcal{H} \subseteq \mathcal{Y}^{\mathcal{X}}$ es agnósticamente 
PAC cognoscible respecto a $Z = \mathcal{X} \times \mathcal{Y}$ y a una función de pérdida 
$l: \mathcal{H} \times Z \rightarrow \mathbb{R}^{+}$ si existe una función 
$m_{\mathcal{H}} : ]0,1[^2\rightarrow \mathbb{N}$ y un algoritmo $A$ verificando que si 
$0 < \epsilon, \delta < 1$, entonces para toda distribución $\mathcal{D}$ sobre $Z$ ejecutando el 
algoritmo para un conjunto de entrenamiento $S\sim \mathcal{D}^m$, con 
$m\ge m_{\mathcal{H}}(\epsilon, \delta)$ se tiene:

\[P_{S\sim \mathcal{D}^m}[L_{\mathcal{D}}(A(S)) \le \underset{h\in \mathcal{H}}{inf} L_{\mathcal{D}}(h) + \epsilon] \ge 1-\delta\]

El algoritmo $A$ devuelve un elemento de $\mathcal{H}$.
\end{definition*}

Notamos desde esta definición tomando una *función de pérdida 0-1*:

\[l_{0-1} (h,(x,y)) := \left\{\begin{array}{ll}
0 & h(x) = y\\
1 & si \quad no
\end{array}\right.\]

equivale a la primera definición que dimos de aprendizaje PAC si asumimos propiedad de factibilidad. Por ello no distinguiremos en el uso de uno u otro concepto, sino que se deducirá de si estamos asumiendo propiedad de factibilidad o no.

Cuando permitimos que el algoritmo $A$ devuelva una función $h \notin \mathcal{H}$, de manera que $h \in \mathcal{H}'$ y $\mathcal{H} \subset \mathcal{H}'$ una clase de funciones donde la función de pérdida es extensible de manera natural, el aprendizaje recibe el nombre de *aprendizaje impropio*. La definición aquí dada se ha hecho para *aprendizaje propio*.


\begin{definition*} \textbf{APAC cognoscible}

Una clase de funciones $H \subseteq 2^X$ es APAC cognoscible sii existen una función 
$m_{H} : ]0,1[^2\rightarrow \mathbb{N}$, llamada complejidad muestral, y un algoritmo 
$A: \underset{m\in \mathbb{N}}{\bigcup} (X\times Y)^m \rightarrow 2^X$ verificando que para todo
$0 < \epsilon, \delta < 1$, para toda distribución $\mathcal{D}$ sobre $X$ y para toda función de 
verdadero etiquetado $f\in H$, dados $m \ge m_H(\epsilon, \delta)$ y $S\sim \mathcal{D}^m$:

\[\mprob \bigg[L_{\mathcal{D},f}(A(S)) > \inf_{h\in H} L_D(h) + \epsilon] < \delta\]
\end{definition*}
