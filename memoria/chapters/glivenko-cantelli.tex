Haremos un breve recorrido por el concepto estadístico de aprendizaje uniforme (clases de Glivenko-Cantelli) y la 
implicación que tiene en teoría PAC. Probaremos que las clases finitas son APAC cognoscibles.

\section{Clases de Glivenko-Cantelli}
\begin{definition}
 Un conjunto de entrenamiento $S$ es $\varepsilon$-representativo de una clase de hipótesis $H$ respecto a una
 distribución $\dist$ y una función de pérdida $l$ (recordemos que $L$ se define en términos de $l$), 
 si $\forall h\in H$ se tiene:
 \[
   |L_S(h) - L_{\dist}(h)| \le \varepsilon
 \]
\end{definition}

Intuitivamente, cualquier $ERM_H$ sobre $S$ sería un buen algoritmo de aprendizaje 
si $S$ es $\varepsilon$ representativo para cualquier $\varepsilon \in ]0,1[$ y cualquier distribución. 
Lo formalizamos con la siguiente proposición.

\begin{fact}
 Si $S$ es $\varepsilon$-representativo de $H$ con respecto a una distribución $\dist$ y una función de 
 pérdida $l$, entonces se tiene:
 \[
   L_{\dist}(ERM_H(S)) \le \inf_{h\in H} L_{\dist}(h) + 2\varepsilon
 \]
 
 \label{fact:epsilon-rep}
\end{fact}

  \begin{proof}
   Sea $A$ un $ERM_H$ en lo que sigue.
   
   $L_{\dist}(A(S)) \le L_S(A(S)) + \varepsilon$ por ser $S$ $\varepsilon$-representativo. Pero por definición de 
   $ERM_H$, $L_S(A(S)) + \varepsilon \le \inf_{h\in H}L_S(h) + \varepsilon$ y aplicando
   $\varepsilon$-representatividad de nuevo:
   \[
     \inf_{h\in H} L_S(h) + \varepsilon \le \inf_{h\in H} L_{\dist}(h) + 2\varepsilon
   \]
   
   En resumen: $L_{\dist}(A(S)) \le \inf_{h\in H} L_{\dist}(h) + 2\varepsilon$, para $A \in ERM_H$ arbitrario.
  \end{proof}


\begin{definition}[Clase de Glivenko-Cantelli]
Decimos que una clase de hipótesis $H$ es de Glivenko-Cantelli, respecto a un dominio $Z$, y a 
una función de pérdida $l$, si existe una función ${m_{H}^{CU}: ]0,1[^2 \rightarrow \mathbb{N}}$ 
verificando que para todo $0 < \delta, \varepsilon < 1$ y para toda distribución $\dist$ sobre $Z$ 
que haga medible a $l$, siendo $S\sim \dist^m$ con $m \ge m_{H}^{CU}(\varepsilon, \delta)$, entonces:
\[
  \mprob [\forall h\in H, |L_S(h) - L_{\dist}(h)| \le \varepsilon] \ge 1-\delta
\]
\end{definition}

Análogamente a lo que ocurría en la definición \ref{def:pac}, podemos considerar $m_H^{CU}$ única, en el sentido de
que para cada $0 < \delta, \varepsilon < 1$, $m_{H}^{CU}(\varepsilon, \delta)$ es el mínimo natural que satisface las 
hipótesis.

\section{Glivenko-Cantelli y APAC cognoscibilidad}

\begin{theorem}
Sea $H$ una clase de hipótesis de Glivenko-Cantelli, respecto a $Z$ y función de pérdida $l$. 
Entonces es APAC cognoscible con cualquier algoritmo $ERM_H$ y complejidad muestral
$m(\varepsilon, \delta) \le m_{H}^{UC} \left(\frac{\varepsilon}{2}, \delta \right)$. 
\label{th:gc-apac}
\end{theorem}

  \begin{proof}
   Sea $A$ un $ERM_H$ arbitrario, y $\dist$ una distribución arbitraria sobre $Z$.
   Fijamos $m \ge m_{H}^{UC} \left(\frac{\varepsilon}{2}, \delta \right)$.

   Sea $S = (z_1, \ldots z_m)$ un conjunto de entrenamiento, verificando que: 
   \begin{equation}
     \forall h\in H, |L_{S}(h)-L_{\dist}(h)| \le \frac{\varepsilon}{2}
     \label{eq:gc-ineq}\tag{$\ast$}
   \end{equation}

   Entonces $S$ es $\varepsilon$ representativa para $\dist$ y $l$, y por proposición \ref{fact:epsilon-rep}:
   \begin{equation}
    L_{\dist}(A(S)) \le \inf_{h\in H} L_{\dist}(h) + \varepsilon
    \label{eq:gc-result}\tag{$\ast$$\ast$}
   \end{equation}

   Pero como $\eqref{eq:gc-ineq}$ ocurre con probabilidad (sobre $S$) mayor o igual a $1-\delta$, entonces 
   $\eqref{eq:gc-result}$ ocurre con probabiliad mayor o igual a $1-\delta$
  \end{proof}
  
  
\begin{fact}
Sea $H$ una clase de hipótesis finita, $Z$ un dominio y sea $l : H \times Z \rightarrow [a,b]$ una función de pérdida.
Entonces $H$ es de Glivenko-Cantelli respecto a $Z$ y $l$ con:
\[
  m_{H}^{CU}(\varepsilon, \delta) \le \left\lceil \frac{log(2|H|/\delta)(b-a)^2}{2\varepsilon^2} \right\rceil
\]
\label{fact:finitas-gc}
\end{fact}
  \begin{proof}
  Fijamos $0 < \delta, \varepsilon < 1$. 
  
  Necesitamos encontrar $m\in \mathbb{N}$ verificando:
  \[
    \mprob [\exists h\in H : |L_S(h) - L_{\dist}(h)| > \varepsilon] < \delta
  \]

  Partimos de la siguiente desigualdad, que usaremos más adelante, obtenida por subaditividad:
  \begin{equation}
  \mprob [\exists h\in H |L_S(h) - L_{\dist}(h)| > \varepsilon] \le 
  \sum_{h \in H} \mprob [|L_S(h) - L_{\dist}(h)| > \varepsilon]
  \label{eqn:gc-subaditividad}\tag{$\ast$}
  \end{equation}

  Fijamos $h \in H$.

  Dado un conjunto de entrenamiento $S = (z_1, \ldots z_m)$, recordamos que 
  $L_{\dist} (h) = \zexpect (l(h,z))$ y que $L_{S}(h) = \frac{1}{m} \sum_{i=1}^m l(h,z_i)$, donde $z_i \sim \dist$ son i.i.d.
  y por tanto:
  \[
    \mexpect (L_S(h)) = \frac{1}{m} \sum_{i=1}^m \underset{z_i\sim \dist} \expect(l(h,z_i)) =  \zexpect (l(h,z)) = L_{\dist} (h)
  \] 
  
  Además, llamando $W_i:Z \rightarrow \mathbb{R}$, $W_i (z_i) = l(h,z_i)$, se verifica que $P[a \le W_i \le b] = 1$
  con las $W_i$ independiente e idénticamente distribuidas. Estamos en condiciones de aplicar la desigualdad \ref{ineq:hoeffding} de Hoeffding.

  Por tanto, por \eqref{eqn:gc-subaditividad}:
  \[
    \mprob \left[\left| \frac{1}{m} \sum_{i=1}^m W_i - L_{\dist} (h) \right| > \varepsilon\right] = 
    P [|L_S(h) - L_{\dist}(h)| > \varepsilon] \le 2e^{-2m \left( \frac{\varepsilon}{b-a} \right)^2}
  \]

  Así:
  \[
    P [\exists h\in H |L_S(h) - L_{\dist}(h)| > \varepsilon] \le 
    |H| 2e^{-2m \left( \frac{\varepsilon}{b-a} \right)^2}
  \]

  Despejando $m$ para que $|H| 2e^{-2m \left( \frac{\varepsilon}{b-a} \right)^2} < \delta$ 
  llegamos al resultado buscado.
  \end{proof}

Recordemos hasta ahora el resultado que habíamos obtenido en el teorema \ref{th:finitas-pac} 
era el carácter PAC cognoscible de las clases de hipótesis finitas. El teorema que sigue generaliza dicho resultado
para cualquier función de pérdida acotada.

\begin{corollary}
Sea $H$ una clase de hipótesis finita, $Z$ un dominio y sea $l : H \times Z \rightarrow [a,b]$ una
función de pérdida. Entonces $H$ es APAC cognoscible con complejidad muestral:
\[
  m_{H}( \varepsilon, \delta ) \le \left\lceil \frac{2 log(2|H|/\delta)(b-a)^2}{\varepsilon^2} \right\rceil
\]

\label{finitas-apac}
\end{corollary}
  
  \begin{proof}
  Estamos en condiciones de aplicar la proposición \ref{fact:finitas-gc}. $H$ sería de Glivenko-Cantelli 
  respecto a $Z,l$. Luego por teorema \ref{th:gc-apac} llegamos al resultado buscaado.
  \end{proof}

Por tanto, si podríamos haber deducido el teorema \ref{th:finitas-pac} a partir del anterior, ¿por qué no probar
directamente este último?. Asintóticamente, para errores muy pequeños $\epsilon \sim 0$, el resultado último nos 
proporciona una cota mucho mayor que \ref{th:finitas-pac}, pues depende de $\frac{1}{\epsilon^2}$, mientras que el primero
dependía de $\frac{1}{\epsilon}$. Para PAC cognoscibilidad el teorema \ref{th:finitas-pac} proporciona un mejor
resultado que este último.