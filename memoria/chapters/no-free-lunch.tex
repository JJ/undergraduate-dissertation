Veremos que dado un algoritmo de aprendizaje PAC no resulta óptimo para aprender todas las distribuciones. 
Lo vemos para clasificación binaria. Consideramos en este capítulo funciones de pérdida $0-1$, puesto que
trabajaremos con paradigma PAC.

\section{Teorema de No Free Lunch}
\begin{lemma}
 Dados $w_1, \ldots w_T \in \mathbb{R}$. Entonces:
 \[
   \min_{i = 1,\ldots ,T} w_i \le \frac{1}{T} \sum_{i=1}^T w_i \le \max_{i = 1,\ldots,T} w_i
 \]
 \label{lemma:nfl-ineqs}
\end{lemma}

\begin{proof}
 Trivial.
\end{proof}

\begin{theorem}[Teorema de No Free Lunch]
Sea un dominio $X$, un conjunto de etiquetas $Y=\{0,1\}$, $H = 2^X$. 
Sea $A: \underset{m\in \mathbb{N}}{\bigcup} (X\times Y)^m \rightarrow 2^X$ un algoritmo. Dado $m \le \frac{|X|}{2}$.
Entonces existe $\dist$ sobre $X\times \{0,1\}$ verificando:
\begin{enumerate}[i]
\item Existe una función $f: X \rightarrow \{0,1\}$ con $L_{\dist}(f)=0$
\item $\mprob \left[ L_{\dist}(A(S)) > \frac{1}{8} \right] \ge \frac{1}{7}$
\end{enumerate}

\label{th:nfl}
\end{theorem}

\begin{proof}
Sea un conjunto de instancias $C=\{x_1, \ldots x_{2m}\}$. Hay $T = 2^{2m}$ posibilidades de etiquetado del conjunto,
esto es, $T$ posibles hipótesis, $f_i: C\rightarrow \{0,1\}$, que vamos a extender a $X$ llamándolas $\bar{f}_i$
de forma que $\bar{f}_{i|C} = f_i$ y $\bar{f}_i(x) = 0 \quad \forall x\in X\setminus C$. 
Vamos a tomar para cada una de ellas una distribución $\dist_i=(2^X, P_i)$ definida por:
\[
  \forall (x,y)\in X \times \{0,1\}, \qquad \underset{z\sim \dist_i}{P_i} [z = (x,y)] = 
  \left\{\begin{array}{ll}
         1/|C| & x \in C, y=f_i(x)\\
         0     & \textrm{si no}
  \end{array}\right. 
\]

Claramente $L_{\dist_i}(f_i) = 0$. Vamos a probar que:
\[
  \exists i\in \{1, \ldots T\} : \quad \mexpecti{i} [L_{\dist_i} (A(S))] \ge \frac{1}{4}
\]

Fijamos $i \in \{1, \ldots T\}$. Hay $k = (2m)^m$ posibles tuplas de tamaño $m$, $S_{j}, j=1, \ldots k$, tomadas 
desde $C$. Siendo $S_j = (x_1, \ldots x_m)$ notamos $S_j^i = ((x_1, f_i(x_1)), \ldots, (x_m, f_i(x_m)))$. Cada 
$S_j^i$ tiene la misma probabilidad de ser nuestro conjunto de entrenamiento (extracción de $m$ valores con 
reemplazamiento desde el conjunto $C$), verificándose:
\[
  \mexpecti{i} [L_{\dist_i} (A(S))] = \frac{1}{k} \sum_{j=1}^k L_{\dist_i} (A(S_j^i))
\]
Recordando que hemos llamado $k=(2m)^m$, $T=2^{2m}$, y usando conmutatividad de sumatorios, se tiene:
\begin{align}
\max_{i \in \{1,\ldots T\}} \frac{1}{k} \sum_{j=1}^{k} L_{\dist_i} (A(S_j^i)) &\underset{\textrm{\ref{lemma:nfl-ineqs}}}{\ge }
       \frac{1}{T} \sum_{i=1}^{T} \frac{1}{k} \sum_{j=1}^{k}  L_{\dist_i} (A(S_j^i)) \nonumber\\
&=     \frac{1}{k} \sum_{j=1}^{k} \frac{1}{T} \sum_{i=1}^{T}  L_{\dist_i} (A(S_j^i)) \nonumber\\
&\underset{\textrm{\ref{lemma:nfl-ineqs}}}{\ge} \min_{j \in \{1, \ldots k\}} \frac{1}{T} \sum_{i=1}^{T}  L_{\dist_i} (A(S_j^i)) 
\label{eqn:nfl-ineqs1}\tag{$\ast$}
\end{align}

Fijamos ahora $j \in \{1,\ldots k\}$:

Sean $\{v_r\}_{r=1}^p$ los elementos de $C$ no presentes en el conjunto de entrenamiento $S_j$. 
Claramente, como $|C|=2m$ y $S_j$ puede tener elementos repetidos, se tiene $p \ge m$. Asimismo:
\[
  L_{\dist_i} (A(S^i_j)) = \frac{1}{|C|} \sum_{x\in C} \mathds{1}_{[A(S^i_j)(x) \neq f_i(x)]} = 
  \frac{1}{2m} \sum_{x \in C} \mathds{1}_{[A(S^i_j)(x) \neq f_i(x)]}
\]

Deducimos:
\begin{align}
\frac{1}{T} \sum_{i=1}^{T}  L_{\dist_i} (A(S_j^i)) &\ge
\frac{1}{T} \sum_{i=1}^{T}  \frac{1}{2m} \sum_{x \in C} \mathds{1}_{[A(S^i_j)(x) \neq f_i(x)]}\nonumber\\
&\ge \frac{1}{T} \frac{1}{2p} \sum_{r=1}^p \sum_{i=1}^{T}  \mathds{1}_{[A(S^i_j)(v_r) \neq f_i(v_r)]} \nonumber\\
&\underset{\textrm{\ref{lemma:nfl-ineqs}}}{\ge} \frac{1}{T} \frac{1}{2} \min_{r} \sum_{i=1}^{T}  \mathds{1}_{[A(S^i_j)(v_r) \neq f_i(v_r)]} 
\label{eqn:nfl-ineqs2}\tag{$\ast$$\ast$}
\end{align}


Dado un $v_r$ cualquiera, $v_r \notin \bigg\{x_k: S_j = (x_1, \ldots, x_m)\bigg\}$. Luego existen 
$f_i, f_{i'}$ que se diferencian justo en el elemento $v_r$. Por tanto, como $S_j^i = S_j^{i'}$,
tendremos que o bien $A(S^i_j)(v_r) \neq f_i(v_r)$ o bien $A(S^{i'}_j)(v_r) \neq f_{i'}(v_r)$:
\begin{equation}
\frac{1}{T} \frac{1}{2} \sum_{i=1}^{T}  \mathds{1}_{[A(S^i_j)(v_r) \neq f_i(v_r)]} = 
  \frac{1}{T} \frac{1}{2} \frac{T}{2} = \frac{1}{4}
\label{eqn:nfl-ineqs3}\tag{$\ast$$\ast$$\ast$}
\end{equation}

Y uniendo \eqref{eqn:nfl-ineqs1}, \eqref{eqn:nfl-ineqs2} y \eqref{eqn:nfl-ineqs3} tendríamos:
\[
  \max_{i \in \{1,\ldots, T\}} \frac{1}{k} \sum_{j=1}^{k} L_{\dist_i} (A(S_j^i)) \ge \frac{1}{4}
\]

Sea $k = \argmax_{i \in \{1,\ldots, T\}} \frac{1}{k} \sum_{j=1}^{k} L_{\dist_i} (A(S_j^i))$.

Si $\dist = \dist_k$ cumple la parte 2 del enunciado del teorema, podemos tomar $f=f_k$ que cumple la primera
parte del enunciado.

Como $L_{\dist} (A(\cdot))$ puede ser vista como una variable aleatoria donde $S \sim \dist^m$ y que toma
valores en $[0,1]$, tenemos que tomando $W = 1-L_{\dist}(A(\cdot))\ge 0$, $a=\frac{7}{8}$, estamos en 
condiciones de aplicar la desigualdad \ref{ineq:markov} de Markov:
\[
  \mprob \left(\frac{1}{8} \ge L_{\dist}(A(S)) \right) \le \frac{3}{4} \cdot \frac{8}{7} = \frac{6}{7}
\]

donde $\mexpect(W) = \mexpect (1 - L_{\dist}(A(\cdot))) = 1 - \mexpect (L_{\dist}(A(\cdot))) \le \frac{3}{4}$

Es decir:
\[
  \mprob \left[ L_{\dist}(A(S)) > \frac{1}{8} \right] \ge \frac{1}{7}
\]
\end{proof}

Como consecuencia del teorema, podemos decir que no hay un algoritmo de aprendizaje óptimo para todas las 
distribuciones, puesto que para un número de puntos $m \le \frac{|X|}{2}$ y la distribución dada por el teorema
anterior, el algoritmo $ERM_H$ con $H = \{f\}$ cometería error nulo, mientras que el que habíamos fijado
obtendría con cierta probabilidad un fallo superior a $\frac{1}{8}$.

\begin{corollary}
 Sea $X$ un dominio con $|X| = \infty$, entonces $2^X$ no es PAC cognoscible.
 \label{cor:P(X)notpac}
\end{corollary}

\begin{proof}
 Supongamos que lo fuese. Sea $A$ el algoritmo que hace a $2^X$ PAC cognoscible. Entonces todo $m\in \mathbb{N}$, como 
 $m\le \frac{|X|}{2}$, el teorema anterior nos diría que existe una distribución $\dist$ verificando: 
 \[
   \mprob \left[ L_{\dist}(A(S)) > \frac{1}{8} \right] \ge \frac{1}{7} \Leftrightarrow 
   \mprob \left[ L_{\dist}(A(S)) \ge \frac{1}{8} \right] \le \frac{6}{7} 
 \]
 
 Luego no puede existir $m_H \left(\frac{1}{8}, \delta\right)$ con $\delta > \frac{6}{7}$.
\end{proof}


\section{Equilibrio error-varianza}
\begin{definition}
 Llamamos error de aproximación a $err_{app} := \inf_{h\in H} L_{\dist}(h)$. LLamamos error de estimación a $err_{est} := 
 L_{\dist} (A(S)) - err_{app}$ para un cierto algoritmo $A$ y conjunto de entrenamiento $S$ arbitrario.
\end{definition}

El equilibrio error-varianza hace referencia a un concepto intuitivo que se desprende de la demostración del teorema
\ref{th:nfl}. La clave de dicha demostración ha sido que estabamos considerando
como conjunto de clasificadores todo el espacio $2^X$. Así, intuitivamente, cuanto más complejo o grande sea el conjunto 
$H$, mejor aproximaremos el verdadero o mejor clasificador, pero más número de muestras necesitaremos y más costoso será
aproximarlo.

Cuando crece $err_{app}$, decrece $err_{est}$ y viceversa. Por tanto, el \emph{machine learning} intenta buscar un equilibrio
entre bondad de los clasificadores que se prueban y la complejidad de la clase a la que pertenecen los mismos.
