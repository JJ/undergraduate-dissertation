En el presente trabajo se han descrito una serie de formalizaciones sobre la teoría del aprendizaje estadístico, que dan
lugar a teoremas tan complejos como el teorema fundamental del aprendizaje PAC o el teorema de No Free Lunch. Ha constituido
un área difícil de estudiar puesto que requiere conocimientos de muchas disciplinas, aunque se ha alcanzado un buen nivel de
asimilación de la teoría presentada.

También se ha conseguido un objetivo fundamental, como era el desarrollo de un software de la temática a tratar, 
dando unas nociones básicas sobre diversos algoritmos noveles que tratan el problema de la clasificación no balanceada.
Se ha comprobado a través de la lectura de \textit{papers} sobre la temática que este es un tema muy estudiado, no sólo a
nivel de informática, sino también de matemática teórica.

Como principales vías futuras que continúen la línea de lo presentado hasta ahora, cabría proponer:

\begin{itemize}
 \item Caracterizar el problema de la clasificación no balanceada a través de los conceptos de dimensión Vapnik Chervonenkis.
  Trabajos como \citep{he2009} aseguran que dicho problema ha sido muy tratado con esta teoría, aunque de manera indirecta,
  a través de modificación de clasificadores SVM. Tras la asimilación de la teoría aquí descrita, sería deseable poder llegar
  a una definición a través de dimensión VC que capturara el concepto de \textit{instancias raras}.
 \item Leer e implementar \citep{pour2015}, un excelente \textit{paper} de la revista \textit{Journal of Machine Learning
  Research} que trata el problema del \textit{oversampling} desde una perspectiva muy matemática, llevándose el espacio
  de dimensiones a un espacio Hilbert, usando funciones kernel, y empleando originales ideas como que los datos generados, al 
  proyectarlos a un espacio de menor dimensión, deberían quedar cerca de los de la clase minoritaria. Este \textit{paper} se ha leído varias
  veces e intentado implementar el algoritmo que propone para este trabajo, pero contiene muchas referencias cruzadas a
  libros y otros artículos de matemáticas y su dificultad quedaba fuera del alcance por el momento. Ahora que por ejemplo
  se dispone de más conocimiento sobre funciones kernel, porque ha sido necesario para comprender el algoritmo PDFOS, sería
  deseable intentar una lectura más profunda y la implementación.
 \item Actualmente \texttt{imbalance} no está alojado en el repositorio CRAN de \R. De aquí a unas semanas se espera enviar
  el software para someterlo a revisión y que pueda ser colgado en dicho repositorio. Esto implicaría que los usuarios no
  tendrían que instalarse una versión en desarrollo desde Github, y que podría instalarse haciendo simplemente desde \R:
  \begin{lstlisting}[numbers=none]
   install.packages("imbalance")
  \end{lstlisting}
  Aparte, supondría un reconocimiento a que el paquete es usable y tiene una documentación apropiada.
\end{itemize}