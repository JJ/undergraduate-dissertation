\usepackage[spanish, es-lcroman]{babel}
\usepackage{amsmath}
\usepackage{amsthm}
\usepackage{fix-cm}
\usepackage{dsfont}
\usepackage{ragged2e}
\usepackage{graphicx}
\definecolor{lightgray}{cmyk}{.30,0,0,.67} % define color using xcolor syntax
\definecolor{midgray}{gray}{0.40}
\usepackage{mdframed} % Add easy frames to paragraphs
\usepackage{lipsum} % For dummy text
\usepackage{xparse} % Add support for \NewDocumentEnvironment
\newmdenv[ % Define mdframe settings and store as leftrule
  linecolor=lightgray,
  topline=false,
  bottomline=false,
  rightline=false,
  skipabove=\topsep,
  skipbelow=\topsep
]{leftrule}

\addtobeamertemplate{block begin}{}{\justifying}
\apptocmd{\frame}{}{\justifying}{}
\apptocmd{\column}{}{\justifying}{}
\let\oldenumerate=\enumerate  
\renewenvironment{enumerate}{\justifying\oldenumerate}{\endlist}
\let\olditem=\item% 
\renewcommand{\item}{\olditem \justifying}


% *************************************************************************************************************
% Pseudocódigo de algoritmos
% *************************************************************************************************************
\usepackage{algorithm}
\usepackage{algorithmic}
\floatname{algorithm}{Algoritmo}
\renewcommand{\listalgorithmname}{Lista de algoritmos}
\renewcommand{\algorithmicrequire}{\textbf{Entrada:}}
\renewcommand{\algorithmicensure}{\textbf{Salida:}}
\usepackage{listings}
\renewcommand{\lstlistingname}{Código}
\definecolor{light-gray}{gray}{0.97}
\lstset{
  language=R,
  literate={<-}{{\boldmath{$\gets$}}}1,
  basicstyle=\small\linespread{1.3}\ttfamily,
  commentstyle=\ttfamily\color{OliveGreen},
  stringstyle=\color{Thistle},
  numbers=left,
  numberstyle=\scriptsize\ttfamily\color{RoyalBlue},
  stepnumber=1,
  numbersep=5pt,
  backgroundcolor=\color{light-gray},
  showspaces=false,
  showstringspaces=false,
  showtabs=false,
  %frame=single,
  tabsize=2,
  captionpos=b,
  breaklines=true,
  breakatwhitespace=false,
  %title=\lstname,
  escapeinside={},
  keywordstyle={},
  morekeywords={}
}



% *************************************************************************************************************
% Entornos
% *************************************************************************************************************
\usepackage{tabularx}
\newcolumntype{C}[1]{>{\centering\let\newline\\\arraybackslash\hspace{0pt}}m{#1}}
\usepackage{float}
\usepackage{wrapfig}

\NewDocumentEnvironment{example}{O{\textbf{Ejemplo:}}}
{\begin{leftrule}\noindent\textcolor{black}{#1}\par}
{\end{leftrule}}

\NewDocumentEnvironment{counterex}{O{\textbf{Contraejemplo:}}}
{\begin{leftrule}\noindent\textcolor{black}{#1}\par}
{\end{leftrule}}
% arg min
\DeclareMathOperator*{\argmin}{arg\,\min }
\DeclareMathOperator*{\argmax}{arg\,\max }

% norma || ||
\newcommand{\norm}[1]{||{#1}||}


\defaultfontfeatures{Ligatures=TeX}

\addto\captionsspanish{%
  \renewcommand{\proofname}%
    {\textit{Sketch} de la prueba}%
}

\newcommand{\img}[2]{
  \begin{center}
  \includegraphics[width=#2\textwidth]{#1}
  \end{center}
}

\newcommand{\imgcaption}[3]{
\begin{figure}[H]
  \begin{center}
  \includegraphics[width=#3\textwidth]{#1}
  \end{center}
  
  \caption[]{#2}
\end{figure}
}

\theoremstyle{definition}

\newtheorem{thm}{theorem}[section]
\newtheorem{theorem}[thm]{Teorema}
\newtheorem*{theorem*}{Teorema}
\newtheorem{fact}[thm]{Proposición}
\newtheorem*{fact*}{Proposición}
\newtheorem{lemma}[thm]{Lema}
\newtheorem*{lemma*}{Lema}
\newtheorem{corollary}[thm]{Corolario}
\newtheorem*{corollary*}{Corolario}
\newtheorem{definition}[thm]{Definición}
\newtheorem*{definition*}{Definición}
\newtheorem{charact}[thm]{Caracterización}
\newtheorem*{charact*}{Caracterización}

\newtranslation[to=Spanish]{Fact}{Proposición}


% *************************************************************************************************************
% Matemáticas
% *************************************************************************************************************
\newcommand{\mprob}{\underset{S\sim \mathcal{D}^m}{P}}
\newcommand{\dosmprob}{\underset{S\sim \mathcal{D}^{2m}}{P}}
\newcommand{\munoprob}{\underset{S_1\sim \mathcal{D}^m}{P}}
\newcommand{\mdosprob}{\underset{S_2\sim \mathcal{D}^m}{P}}
\newcommand{\mmprob}{\underset{\begin{subarray}{c} 
                                S_1 \sim \mathcal{D}^{m} \\ 
                                S_2 \sim\mathcal{D}^m 
                                \end{subarray}}{P}}
\newcommand{\prob}{\underset{x\sim \mathcal{D}}{P}}
\newcommand{\zprob}{\underset{z\sim \mathcal{D}}{P}}
\newcommand{\dist}{\mathcal{D}}
\newcommand{\expect}{\mathbb{E}}
\newcommand{\zexpect}{\underset{z\sim \mathcal{D}}{\expect}}
\newcommand{\mexpecti}[1]{\underset{S\sim \mathcal{D}_{#1}^m}{\mathbb{E}}}
\newcommand{\mexpect}{\underset{S\sim \mathcal{D}^m}{\mathbb{E}}}
\newcommand{\dosmexpect}{\underset{S\sim \mathcal{D}^{2m}}{\mathbb{E}}}
\newcommand{\ppos}{P^{+}}
\newcommand{\pneg}{P^{-}}
\newcommand{\npos}{N^{+}}
\newcommand{\nneg}{N^{-}}
\newcommand{\spos}{S^{+}}
\newcommand{\sneg}{S^{-}}
\newcommand{\Hclass}{\lfloor H \rfloor_{\dist}}
\newcommand{\algcomp}[1]{\mathcal{O}\left(#1\right)}

