\subsection{Introducción a la probabilidad}
 \begin{frame}\frametitle{$\sigma$ álgebras, anillos y semianillos}
  \begin{definition}
   $\Sigma \subseteq 2^X$ es $\sigma$-álgebra de conjuntos sobre $X$ si se verifica:
   
   \begin{enumerate}[i]
    \item $X \in \Sigma$
    \item Es cerrada para complementarios: $A\in \Sigma$, entonces $A^c = X\setminus A \in \Sigma$
    \item Es cerrada para uniones numerables: $\{A_n\}_{n\in\mathbb{N}} \subseteq \Sigma$, entonces 
    $\underset{n \ge 1}{\bigcup} A_n \in \Sigma$
   \end{enumerate}
   
   Si $\emptyset \in \Sigma$, es anillo si:
   \begin{enumerate}[i]
    \item $A,B \in R$ entonces $A\cup B \in \Sigma$
    \item $A,B \in R$ entonces $A\setminus B \in \Sigma$
   \end{enumerate}

   y es semianillo si cumple:
   \begin{enumerate}[i]
    \item $A,B \in \Sigma$ entonces $A\cap B \in \Sigma$
    \item $A,B \in \Sigma$ entonces existen $A_1, \ldots, A_n \in \Sigma$ verificándose $A\setminus B = \sum_{i=1}^n A_i$
   \end{enumerate}
  \end{definition}

  \begin{fact}
  Toda $\sigma$-álgebra es anillo. Todo anillo es semianillo.
  \end{fact}
 \end{frame}
 
 \begin{frame}\frametitle{Medidas, medidas exteriores, probabilidades}
 \begin{definition}[Medida y medida exterior]
  Sea $\mathcal{A} \subseteq 2^X$. Sea $\mu: \mathcal{A} \rightarrow [0, +\infty]$ con $\mu(\emptyset) = 0$. Entonces:
  
  \begin{itemize}
   \item $\mu$ es medida si es $\sigma$-aditiva, esto es, dados $A_n \in \mathcal{A}$ con 
   \[
    \sum_{i=1}^n A_i \in \mathcal{A} \Rightarrow \mu\left(\sum_{i=n}^{+\infty} A_n \right)= \sum_{i=n}^{+\infty} \mu(A_n)
   \]
  
   \item Si $\mathcal{A}=2^X$, $\mu$ es medida exterior sobre $X$ si:
  
    \begin{enumerate}[i]
     \item Monotonía: $A\subseteq B$ entonces $\mu(A) \le \mu(B)$
     \item $\sigma$-subaditividad: $\mu \left(\bigcup_{n=1}^{+\infty} A_n \right) \le \sum_{n=1}^{+\infty} \mu (A_n)$
    \end{enumerate}
  \end{itemize}
 \end{definition}

 \vspace{-1em}
 \begin{definition}[Espacio de probabilidad]
  Sea $(X, \Sigma, P)$, con $\Sigma$ $\sigma$-álgebra sobre $X$ y $P:\Sigma \rightarrow [0, +\infty]$ medida. 
  Entonces lo llamamos espacio de probabilidad sii $P(\Sigma)\subseteq [0,1]$.
 \end{definition}
 
 \begin{definition}[Distribución de probabilidad]
  Sea $(X, \Sigma, P)$ espacio de probabilidad. Llamamos a la tupla $\dist = (\Sigma,P)$ distribución sobre $X$. 
  Si $\dist$ es distribución sobre $X$, lo notamos $x\sim \mathcal{D}$.
 \end{definition}
 \end{frame}
 
 \begin{frame}\frametitle{Teorema de extensión de Carathéodory}
  \begin{definition}
  $\Sigma(\mu^\ast) = \{A\subseteq X: \mu^\ast(T) = \mu^\ast(T\cap A) + \mu^\ast(T\cap A^c), \forall T\subseteq X\}$
  \end{definition}
  \begin{theorem}
   Sea $\mu^\ast : 2^X \rightarrow [0, +\infty]$ medida exterior sobre $X$. Entonces $\Sigma(\mu^\ast)$
   es una $\sigma$-álgebra sobre $X$ y $\mu^\ast_{|\Sigma(\mu^\ast)}$ es una medida sobre $\Sigma(\mu^\ast)$
  \end{theorem}
 
  \begin{theorem}[Teorema de extensión de Carathéodory]
   Sea $S \subseteq 2^X$ un semianillo, $\mu:S \rightarrow [0,+\infty]$ medida en $S$. Entonces existe
   una medida $\overline{\mu}:\Sigma(S) \rightarrow [0,+\infty]$ verificando $\overline{\mu}_{|S} = \mu$
   \label{th:caratheodory}
  \end{theorem}
 
 \begin{proof}
  \begin{itemize}
   \item LLamar $R=(S)$. Probar $\Sigma(S) = \Sigma(R)$.
  
   \item Si existiese $\overline{\mu}$ debería verificar, $\overline{\mu}(T) \le \sum_{n=1}^{+\infty} \overline{\mu}(A_n)$, 
    con $T \subseteq \bigcup_{n\ge 1} A_n$.
  
   \item Definir $\mu^\ast(T):= \inf\left\{\sum_{n=1}^{+\infty} \mu(A_n): \bigcup_{n\ge 1} A_n \supseteq T,
                   A_n\in R \quad \forall n\in\mathbb{N}\right\}$.
   \item Probar que $\mu^\ast$ es medida exterior sobre $X$ y usar el teorema anterior.
  \end{itemize}
 \end{proof}
 \end{frame}  
 
 \begin{frame}\frametitle{Espacio de probabilidad producto}
  \begin{theorem}
   Sean $(X_i, \Sigma_i, P_i)$ con $i=1,\ldots, n$ espacios de probabilidad. Entonces existe un espacio probabilidad
   producto.
   \[(X_1 \times \ldots \times X_n, \Sigma_1 \otimes \ldots \otimes \Sigma_n, P)\]
   donde:
   \begin{enumerate}[i]
   \item $\Sigma_1 \otimes \ldots \otimes \Sigma_n = \Sigma\left(\{A_1 \times \ldots \times A_n: A_i\in \Sigma_i\}\right)$
   \item $P(A_1 \times \ldots \times A_n) = \prod_{i=1}^n P_i(A_i)$, con $A_i\in \Sigma_i$ arbitrarios.
  \end{enumerate}
  \end{theorem}
  
  Se puede probar que el espacio probabilístico producto sólo hay uno, pero no usaremos este hecho, aunque conviene tenerlo
  presente. Si $\dist = \dist_1 = \ldots = \dist_n$, llamaremos $\dist^n$ al único espacio de probabilidad producto.
 \end{frame}
 
 \begin{frame}\frametitle{Desigualdades de concentración}
 \begin{lemma}[Desigualdad de Markov]
  Sea $W \ge 0$ una variable aleatoria. Entonces para todo $a > 0$ se verifica:
  \[
    P[W \ge a] \le \frac{\expect(W)}{a}
  \]
  \label{ineq:markov}
 \end{lemma}
  
  \begin{lemma}[Desigualdad de Hoeffding]
   Sean $W_1, \ldots, W_m$ variables aleatorias i.i.d. (independiente e idénticamente distribuidas), tales que 
   $a_i \le W_i \le b_i$, y sea $\overline{W} = \frac{1}{m} \sum_{i=1}^m W_i$. Entonces dado $\varepsilon > 0$ arbitrario:
   \[
     P\left[\left| \overline{W} - \expect(\overline{W}) \right| \ge \varepsilon \right] \le 2e^{-2m^2 \frac{\varepsilon^2}{\sum_{i=1}^m (b_i-a_i)^2}}
   \]
   
   En el caso particular $a_i = a$ y $b_i = b$ para todo $i=1, \ldots, m$:
   \[
     P\left[\left| \overline{W} - \expect(\overline{W}) \right| \ge \varepsilon \right] \le 2e^{-2m \left(\frac{\varepsilon}{b-a}\right)^2}
   \] 
   \label{ineq:hoeffding}
  \end{lemma}
 \end{frame}