\subsection{Introducción}
 \begin{frame}\frametitle{Introducción}
 \begin{definition}[Instancias positivas, negativas]
 Llamamos:
 \begin{itemize}
  \item Instancias positivas a $Z^{+} = \{(x,y)\in Z: y=1\}$.
  \item Instancias negativas a $Z^{-} = \{(x,y)\in Z: y=-1\}$.
 \end{itemize}
 Una notación equivalente se aplica a $S$, conjunto de entrenamiento. Además llamamos $\ppos = P(Z^{+})$ y $\pneg = P(Z^{-})$.
 \end{definition}

 \begin{definition}[Desbalanceo entre clases]
  Decimos que un problema de clasificación binaria es desbalanceado (entre clases) si se verifica que tomando 
  $\nneg = |S^{-}|$ y $\npos = |S^{+}|$, entonces $\npos < \nneg$.
  
  Llamamos ratio de desbalanceo a $r(S)=\frac{\npos}{\nneg}$. Normalmente $r(S) \ll 1$.
 \end{definition}

 Para ciertos conjuntos de datos, donde el número de instancias positivas con respecto a las negativas es muy pequeño, un clasificador podría simplemente
 etiquetar todas las instancias como negativas, y obtener un error global pequeño, pero no acertaría en ninguna instancia
 positiva. Si los datasets empleados representaran enfermedades, por ejemplo,
 nos interesa encontrar mecanismos para disminuir el error sobre la clase positiva (la de las instancias correspondientes
 a la enfermedad).
\end{frame}

\begin{frame}\frametitle{Tipos de desbalanceo}
\begin{columns} 
 \begin{column}{0.55\textwidth}
  El desbalanceo puede ser:
  
  \begin{enumerate}[i]
   \item \textbf{Intrínseco}, cuando $\ppos < \pneg$.

   \item \textbf{Extrínseco}, si $\ppos \ge \pneg$ pero sin embargo para nuestro conjunto de entrenamiento $S$, 
   tenemos que $\npos < \nneg$.
  \end{enumerate}
  
  \justifying
  Se dice que existe desbalanceo \textit{intra clases} cuando además la clase minoritaria está repartida en varias 
  regiones no conexas, a saber $\spos \supset A_1, \ldots, A_m$ donde $|A_1| \le \ldots \le |A_m|$, con algún menor o igual estricto.
  Esto dificulta aún más la correcta clasificación de dichas instancias, porque aparecen zonas con instancias positivas 
  pobremente representadas. A estas zonas de baja representación las llamamos instancias raras.
 \end{column}
 
 \begin{column}{0.45\textwidth}
  \img{../memoria/imgs/desbalanceo.png}{1.1}
 \end{column}
\end{columns}
\end{frame}

\begin{frame}\frametitle{Aprendizaje con desbalanceo}
\par\textbf{Técnicas de aprendizaje con desbalanceo}
\begin{itemize}
 \item \textbf{\textit{Oversampling/Undersampling}}. Consiste en tomar $E$ instancias y entrenar con 
 $S' = S\cup E$ ($E$ positivas) o $S' = S\setminus E$ ($E$ negativas) resp.
 
 Tanto el \textit{oversampling} como el \textit{undersampling} se realiza de manera que $r(S') > r(S)$. Se pueden usar además
 técnicas de limpieza, para definir mejor los bordes de las zonas que conectan a ambas clases, de manera que mejoremos el error
 en la clasificación.
 
 \item \textbf{Aprendizaje \textit{cost sensitive}}. El \textit{framework} de aprendizaje \textit{cost sensitive} implica que no
 a todas las instancias le asignamos el mismo error al aprender.
\end{itemize}
\medskip

\par\textbf{Medidas de la bondad del aprendizaje}
\begin{columns}
 \begin{column}{0.5\textwidth}
  \begin{table}[H]
    \centering
    \begin{tabular}{C{2cm}|C{2cm}}
    $VP$ & $FP$\\
    \hline
    $FN$ & $VN$\\
    \end{tabular}
    \caption{Matriz de confusión}
  \end{table}
 \end{column}
 
 \begin{column}{0.5\textwidth}
  \begin{definition}[Sensibilidad]
   Se define la sensibilidad como $s = \frac{VP}{\npos}$ 
  \end{definition}
  
  Hay otras medidas como el $F$-score, el $AUC$, \ldots
 \end{column}
\end{columns}
\end{frame}

