\subsection{Experimentación con \textit{small disjuncts}}
\fontsize{8pt}{0}\selectfont
\begin{frame}\frametitle{Experimentación}
 \vspace{2em}
 Los clasificadores intentan aprender a partir de una clase 
 creando reglas (conceptos) disjuntas que agrupen a instancias. Debido a la infrarrepresentación y al 
 desbalanceo \textit{intra clases}, podemos tener reglas que cubran una pequeña porción de la clase 
 positiva (\textit{small disjuncts}).
  \begin{table}[H]
  \centering
  \begin{tabular}{rrrrrr}
  \hline
  & none & mwmote & wracog & rwo & pdfos \\ 
  \hline
  iris0 & 0.00 & 0.00 &  & 0.00 & 0.00 \\ 
  newthyroid1 & 1.67 & 1.33 &  & 1.67 & 1.00 \\ 
  wisconsin & 27.67 & 62.33 & 36.00 & 65.67 & 65.67 \\ 
  \hline
  \end{tabular}
  \caption{Media de \textit{small disjuncts} tras \textit{oversampling}}
  \end{table}

  \vspace{-2em}
  
  \begin{table}[H]
  \centering
  \begin{tabular}{rrrrrr}
  \hline
  & none & mwmote & wracog & rwo & pdfos \\ 
  \hline
  iris0 & 25.00 & 30.33 &  & 30.33 & 30.33 \\ 
  newthyroid1 & 19.89 & 30.00 &  & 20.40 & 23.40 \\ 
  wisconsin & 6.74 & 3.88 & 5.68 & 3.65 & 3.65 \\ 
  \hline
  \end{tabular}
  \caption{Tamaño medio de coberturas tras \textit{oversampling}}
  \end{table}
  
  \vspace{-2em}
  
  \begin{table}[H]
  \centering
  \begin{tabular}{rrrrrr}
  \hline
  & none & mwmote & wracog & rwo & pdfos \\ 
  \hline
  ecoli1 & 0.76 & 1.00 &  & 0.96 &  \\ 
  glass0 & 0.97 & 0.98 &  & 1.00 &  \\ 
  haberman & 0.54 & 0.85 &  & 0.76 & 0.68 \\ 
  iris0 & 1.00 & 1.00 &  & 1.00 & 1.00 \\ 
  newthyroid1 & 0.97 & 1.00 &  & 0.99 & 1.00 \\ 
  wisconsin & 1.00 & 1.00 & 1.00 & 1.00 & 1.00 \\ 
  yeast4 & 0.33 & 1.00 &  & 0.99 &  \\ 
  \hline
  \end{tabular}
  \caption{Media de sensibilidad tras \textit{oversampling}}
  \end{table}
\end{frame}



\begin{frame}\frametitle{Experimentación}
  \begin{table}[H]
  \centering
  \begin{tabular}{rrrrrr}
  \hline
  & none & mwmote & wracog & rwo & pdfos \\ 
  \hline
  iris0 & 0.00 & 0.00 &  & 0.00 & 0.00 \\ 
  newthyroid1 & 1.67 & 1.33 &  & 2.33 & 1.00 \\ 
  wisconsin & 27.67 & 62.33 & 34.33 & 65.00 & 63.67 \\ 
  \hline
  \end{tabular}
  \caption{Media de \textit{small disjuncts} tras \textit{oversampling} y filtrado}
  \end{table}

  \vspace{-2em}
  
  \begin{table}[H]
  \centering
  \begin{tabular}{rrrrrr}
   \hline
   & none & mwmote & wracog & rwo & pdfos \\ 
   \hline
   iris0 & 25.00 & 30.33 &  & 30.33 & 30.33 \\ 
   newthyroid1 & 19.89 & 28.83 &  & 21.19 & 23.85 \\ 
   wisconsin & 6.74 & 3.88 & 5.78 & 3.67 & 3.71 \\
   \hline
  \end{tabular}
  \caption{Tamaño medio de coberturas tras \textit{oversampling} y filtrado}
  \end{table}
  
  \vspace{-2em}
  
  \begin{table}[H]
  \centering
  \begin{tabular}{rrrrrr}
  \hline
  & none & mwmote & wracog & rwo & pdfos \\ 
  \hline
  ecoli1 & 0.76 & 0.98 &  & 0.96 &  \\ 
  glass0 & 0.97 & 0.98 &  & 0.99 &  \\ 
  haberman & 0.54 & 0.83 &  & 0.85 & 0.87 \\ 
  iris0 & 1.00 & 1.00 &  & 1.00 & 1.00 \\ 
  newthyroid1 & 0.97 & 1.00 &  & 0.99 & 1.00 \\ 
  wisconsin & 1.00 & 1.00 & 1.00 & 1.00 & 1.00 \\ 
  yeast4 & 0.33 & 0.99 &  & 0.99 &  \\ 
  \hline
  \end{tabular}
  \caption{Media de sensibilidad tras \textit{oversampling} y filtrado}
  \end{table}
  
\end{frame}
